%
% "The UCD CASL SenseTile System", for IEEE Pervasive Computing
% Joseph R. Kiniry, Vieri del Bianco, Dragan Stosic
% $Id: paper.tex 2795 2007-09-23 19:35:53Z dmz $
%

\documentclass[times,10pt,twocolumn]{article} 
\usepackage{latex8}
\usepackage{times}

\usepackage{ifpdf}
\usepackage{a4wide}
\usepackage{pdfsync}

\ifpdf
\usepackage[pdftex]{graphicx}
\else
\usepackage{graphicx}
\fi

%\usepackage{todonotes}

% \usepackage{html}
% \usepackage{url}
\usepackage{xspace}
%\usepackage{doublespace}
\usepackage{tabularx}
\usepackage{epsfig}
\usepackage{amsmath}
\usepackage{amsfonts}
\usepackage{amssymb}
\usepackage{eucal}
\usepackage{stmaryrd}
% \ifpdf
% \usepackage[centredisplay]{diagrams}
% \else
% \usepackage[centredisplay,PostScript=dvips]{diagrams}
% \fi
\usepackage{float}

\ifpdf
\usepackage[pdftex,bookmarks=false,a4paper=false,
            plainpages=false,naturalnames=true,
            colorlinks=true,pdfstartview=FitV,
            linkcolor=blue,citecolor=blue,urlcolor=blue,
            pdfauthor="Joseph R. Kiniry and Vieri del Bianco and Dragan Stosic"]{hyperref}
\else
\usepackage[dvips,linkcolor=blue,citecolor=blue,urlcolor=blue]{hyperref}
\fi

\newcommand{\tablesize}{\footnotesize}
\newcommand{\eg}{e.g.,\xspace}
\newcommand{\ie}{i.e.,\xspace}
\newcommand{\etc}{etc.\xspace}
\newcommand{\myhref}[2]{\ifpdf\href{#1}{#2}\else\htmladdnormallinkfoot{#2}{#1}\fi}
%\newcommand{\myhref}[2]{\emph{#2}}
\newcommand{\todo}{\textbf{TODO:}}

%---------------------------------------------------------------------
% New commands, macros, \etc
%---------------------------------------------------------------------

%% \input{kt}

%=====================================================================

\begin{document}

\title{The UCD CASL SenseTile System}

\author{Joseph R. Kiniry, Vieri del Bianco, Dragan Stosic\\
UCD CASL: Complex and Adaptive Systems Laboratory and\\
School of Computer Science and Informatics,\\
University College Dublin,\\
Belfield, Dublin 4, Ireland,\\
kiniry@acm.org, vieri.delbianco@gmail.com, dragan.stosic@gmail.com\\
}

\maketitle

%======================================================================
%\thispagestyle{empty}
\begin{abstract}
  
\todo{\emph{We probably have only 8 pages, in the IEEE format, if this is going to
\myhref{http://ieeexplore.ieee.org/xpl/RecentIssue.jsp?punumber=7756}{IEEE Pervasive Computing}}.}

An abstract.

\end{abstract}

%======================================================================
\Section{Introduction}

An introduction.

%=====================================================================
\Section{Conclusion}

A conclusion.

%======================================================================
%% \nocite{ex1,ex2}
\bibliographystyle{latex8}
\bibliography{extra,%
              abbrev,%
              ads,%
              category,%
              complexity,%
              hypertext,%
              icsr,%
              knowledge,%
              languages,%
              linguistics,%
              meta,%
              metrics,%
              misc,%
              modeling,%
              modeltheory,%
              reuse,%
              rewriting,%
              softeng,%
              specification,%
              ssr,%
              technology,%
              theory,%
              web,%
              upcoming,%
              upcoming_conferences,%
              conferences,%
              workshops,%
              verification,%
              escjava,%
              jml,%
              nijmegen}

%======================================================================
% Fin

\end{document}

%%% Local Variables: 
%%% mode: latex
%%% eval: 
%%% TeX-master: t
%%% End: 
