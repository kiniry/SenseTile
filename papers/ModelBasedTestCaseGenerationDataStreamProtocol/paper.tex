%
% "The UCD CASL SenseTile System", for IEEE Pervasive Computing
% Joseph R. Kiniry, Vieri del Bianco, Dragan Stosic
% $Id: paper.tex 2795 2007-09-23 19:35:53Z dmz $
%

\documentclass{article} 
\usepackage{times}

\usepackage{ifpdf}
\usepackage{a4wide}
\usepackage{pdfsync}

\ifpdf
\usepackage[pdftex]{graphicx}
\else
\usepackage{graphicx}
\fi

%\usepackage{todonotes}

% \usepackage{html}
% \usepackage{url}
\usepackage{xspace}
%\usepackage{doublespace}
\usepackage{tabularx}
\usepackage{epsfig}
\usepackage{amsmath}
\usepackage{amsfonts}
\usepackage{amssymb}
\usepackage{eucal}
\usepackage{stmaryrd}
% \ifpdf
% \usepackage[centredisplay]{diagrams}
% \else
% \usepackage[centredisplay,PostScript=dvips]{diagrams}
% \fi
\usepackage{float}

\ifpdf
\usepackage[pdftex,bookmarks=false,a4paper=false,
            plainpages=false,naturalnames=true,
            colorlinks=true,pdfstartview=FitV,
            linkcolor=blue,citecolor=blue,urlcolor=blue,
            pdfauthor="Joseph R. Kiniry and Vieri del Bianco and Dragan Stosic"]{hyperref}
\else
\usepackage[dvips,linkcolor=blue,citecolor=blue,urlcolor=blue]{hyperref}
\fi

\newcommand{\tablesize}{\footnotesize}
\newcommand{\eg}{e.g.,\xspace}
\newcommand{\ie}{i.e.,\xspace}
\newcommand{\etc}{etc.\xspace}
\newcommand{\myhref}[2]{\ifpdf\href{#1}{#2}\else\htmladdnormallinkfoot{#2}{#1}\fi}
%\newcommand{\myhref}[2]{\emph{#2}}
\newcommand{\todo}{\textbf{TODO:}}
\newcommand{\ST}{\emph{SenseTile}\xspace}
\newcommand{\STs}{\emph{SenseTiles}\xspace}
\newcommand{\STS}{\emph{SenseTile Simulator}\xspace}
\newcommand{\datastore}{\STs Scientific Datastore\xspace}
\newcommand{\computefarm}{The \STs Scientific Compute Farm\xspace}
\newcommand{\computefarmlong}{UCD CASL \STs Software and Data Compute Server Farm\xspace}
\newcommand{\sensorfarm}{The \STs\xspace}

%---------------------------------------------------------------------
% New commands, macros, \etc
%---------------------------------------------------------------------

%% \input{kt}

%=====================================================================

\begin{document}

\title{Deriving an Executable Model of a Data Stream Protocol from its
Specification}

\author{Joseph R. Kiniry, Vieri del Bianco, Dragan Stosic\\
UCD CASL: Complex and Adaptive Systems Laboratory and\\
School of Computer Science and Informatics,\\
University College Dublin,\\
Belfield, Dublin 4, Ireland,\\
kiniry@acm.org, vieri.delbianco@gmail.com, dragan.stosic@gmail.com\\
}

\maketitle

%======================================================================
%\thispagestyle{empty}
\begin{abstract}

\begin{verbatim}
OBJECTIVE: 
Specifications used as a blueprint for system development.
Specifications to generate test suite to validate the developed system.
Specifications to validate simulator of the developed system.
Apply to data stream processing: verify the stream is parsed correctly, 
verify the parsed stream output stream can be simulated.

METHODS: 
JML model.
JML tools to generate JUnit test suite.
JUnit test suite applied to developed system, to verify its correctness.
Junit test suite applied to simulator, to verify its correctness.

RESULTS: 
Proof developed system same behaviour as simulator.
Proof developed system and simulator verify the specifications.

CONCLUSIONS:
Specifications serves two purposes, and indirectly a third one.
\end{verbatim}

  Sensor packages typically communicate (control sensors and report
  sensor measurements) with a streaming packet-based network protocol.
  The \ST device is an example of such a sensor package, as it
  communicates with an arbitrary computer system (running Linux or
  Windows, at the moment) over a USB bus using a streaming
  packet-based protocol.  This sensor platform is a custom-designed,
  but inexpensive, sensor platform (called the \ST) paired with
  general-purpose small-scale compute nodes, including everything from
  low-power PDAs to powerful touchscreen-based portable computers.
  There are over a dozen sensors on the \ST itself, and new sensors
  can be added by adding them to the USB bus.  A third-party low-level
  C driver, provided by the USB chip vendor, creates and parses
  generic USB communication packets, and a custom medium-level Java
  driver creates and parses \ST streaming communication packets.  This
  latter Java driver is formally specified, validated with rigorous
  unit testing, and verified using extended static checking.  However,
  because of this process and combination of technologies, using this
  specification we are able to both communicate with a real \ST
  \textbf{and} generate a formally specified, executable, functional
  model-based specification for the protocol itself written in the
  Java Modeling Language.  This protocol model is itself an executable
  stream simulator that can be used to develop new \ST user
  applications without having the \ST hardware at all.  Consequently,
  from this single specification, we can test that the \ST hardware
  communicates as expected, test that our \ST simulator communicates
  as expected, and simulate the \ST protocol stream itself.

\end{abstract}

%======================================================================
\Section{Introduction}

The \textbf{UCD CASL SenseTile System} is a large-scale, general-purpose
sensor system installed at the University College Dublin in Dublin,
Ireland.  \textbf{Our facility provides a capability---unique in the world in
terms of its scale and flexibility---for performing in-depth
investigation of both the specific and general issues in large-scale
sensor networking.}

This system integrates a sensor platform, a datastore, and a compute
farm.  The sensor platform is a custom-designed, but inexpensive,
sensor platform (called the \ST) paired with general-purpose
small-scale compute nodes, including everything from low-power PDAs to
powerful touchscreen-based portable computers.  There are over a dozen
sensors on the \ST itself, and new sensors can be added by adding them
to the USB bus.



%=====================================================================
\Section{Conclusion}

A conclusion.

%======================================================================
%% \nocite{ex1,ex2}
\bibliographystyle{latex8}
\bibliography{extra,%
              abbrev,%
              ads,%
              category,%
              complexity,%
              hypertext,%
              icsr,%
              knowledge,%
              languages,%
              linguistics,%
              meta,%
              metrics,%
              misc,%
              modeling,%
              modeltheory,%
              reuse,%
              rewriting,%
              softeng,%
              specification,%
              ssr,%
              technology,%
              theory,%
              web,%
              upcoming,%
              upcoming_conferences,%
              conferences,%
              workshops,%
              verification,%
              escjava,%
              jml,%
              nijmegen}

%======================================================================
% Fin

\end{document}

%%% Local Variables: 
%%% mode: latex
%%% eval: 
%%% TeX-master: t
%%% End: 
