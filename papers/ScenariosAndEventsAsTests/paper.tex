%
% "Formal Refinement between High-level Specifications and Validation" for UNKNOWN09.
% Joseph R. Kiniry
% $Id: paper.tex 2795 2007-09-23 19:35:53Z dmz $
%

\documentclass{article} 
\usepackage{times}
\usepackage{multirow}
\usepackage{ifpdf}
\usepackage{a4wide}
\usepackage{pdfsync}

\ifpdf
\usepackage[pdftex]{graphicx}
\else
\usepackage{graphicx}
\fi

%\usepackage{todonotes}

% \usepackage{html}
% \usepackage{url}
\usepackage{xspace}
%\usepackage{doublespace}
\usepackage{tabularx}
\usepackage{epsfig}
\usepackage{amsmath}
\usepackage{amsfonts}
\usepackage{amssymb}
\usepackage{eucal}
\usepackage{stmaryrd}
% \ifpdf
% \usepackage[centredisplay]{diagrams}
% \else
% \usepackage[centredisplay,PostScript=dvips]{diagrams}
% \fi
\usepackage{float}

\ifpdf
\usepackage[pdftex,bookmarks=false,a4paper=false,
            plainpages=false,naturalnames=true,
            colorlinks=true,pdfstartview=FitV,
            linkcolor=blue,citecolor=blue,urlcolor=blue,
            pdfauthor="Joseph R. Kiniry"]{hyperref}
\else
\usepackage[dvips,linkcolor=blue,citecolor=blue,urlcolor=blue]{hyperref}
\fi

\newcommand{\tablesize}{\footnotesize}
\newcommand{\eg}{e.g.,\xspace}
\newcommand{\ie}{i.e.,\xspace}
\newcommand{\etc}{etc.\xspace}
\newcommand{\myhref}[2]{\ifpdf\href{#1}{#2}\else\htmladdnormallinkfoot{#2}{#1}\fi}
%\newcommand{\myhref}[2]{\emph{#2}}
\newcommand{\todo}{\textbf{TODO:}}
\newcommand{\name}{\textbf{IB2JML}}
%---------------------------------------------------------------------
% New commands, macros, \etc
%---------------------------------------------------------------------

%% \input{kt}

%=====================================================================

\begin{document}

\title{Formal Refinement between\\
High-level Specifications and Validation}

\author{Joseph R. Kiniry\\
UCD CASL: Complex and Adaptive Systems Laboratory and\\
School of Computer Science and Informatics,\\
University College Dublin,\\
Belfield, Dublin 4, Ireland,\\
kiniry@acm.org\\
}

\maketitle

%======================================================================
%\thispagestyle{empty}
\begin{abstract}
  
  An abstract.

\end{abstract}

%======================================================================
\section{Introduction}

The typical requirements document for a complex software project
consists of hundreds or thousands of labeled English sentences written
in a Microsoft Word document.  There is little-to-no relationship
between this document's contents and the architecture and
implementation of the system they purport to describe.  Checking that
a system conforms to the requirements is a labor-intensive, expensive,
slow, and manual process performed by some poor engineer that drew the
short straw.  Moreover, like the vast majority of architecture
descriptions using languages like UML, the system and its description
are not kept in sync, so as the implementation moves on, the
requirements are tossed to the wayside.

In this paper we describe a a formal refinement between high-level
specifications and a system's implementation via its validation.  The
high-level specifications that we focus on include scenarios (aka
use-cases), requirements, features (a la software product lines), and
events, all of which are expressed in the Extended Business Object
Notation (EBON) specification language.  Validation includes
(sub)system, regression, and unit testing and static verification,
using the Java Modeling Language (JML).  This refinement is expressed
as a relation on types which is realizable as a adjoint pair of
computations.  Consequently, one can use this formal refinement to (1)
check that an implementation and its architecture conform to its
high-level specification, (2) automatically update an architecture and
implementation as their high-level specifications change, \emph{and}
(3) automatically update high-level specifications as their
architecture or implementation evolve.

Our formal refinement is mechanically formally specified in
higher-order logic in the PVS theorem prover and various appropriate
metatheoretical properties about it are shown to hold.  Tool support
for checking and updates is provided by the \name tool, an Eclipse
plugin that realizes this formal refinement between EBON and JML.

%=====================================================================
\section{Conclusion}

A conclusion.

%% \nocite{ex1,ex2}
\bibliographystyle{plain}
\bibliography{extra,%
              abbrev,%
              ads,%
              category,%
              complexity,%
              hypertext,%
              icsr,%
              knowledge,%
              languages,%
              linguistics,%
              meta,%
              metrics,%
              misc,%
              modeling,%
              modeltheory,%
              reuse,%
              rewriting,%
              softeng,%
              specification,%
              ssr,%
              technology,%
              theory,%
              web,%
              upcoming,%
              upcoming_conferences,%
              conferences,%
              workshops,%
              verification,%
              escjava,%
              jml,%
              nijmegen}

%======================================================================
% Fin

\end{document}

%%% Local Variables: 
%%% mode: latex
%%% eval: 
%%% TeX-master: t
%%% End: 
