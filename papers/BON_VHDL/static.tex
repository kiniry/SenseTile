\begin{center}
    \begin{tabular}{|c|c|l|}
        \hline
        \multicolumn{3}{|c|}{Static mapping}\\ \hline
    	\hline
    	VHDL-AMS & BON & Description \\ \hline
    	\ducs & \Sc & 
	\begin{minipage}[c]{0.5\linewidth}
                 \smallskip
    		VHDL-AMS allows hierarchical structure modeling, 
    		meaning that subsistems (\ducs), can be developed independently.
    		As an top-level description (informal/formal), \Sc connect all the \ducs
    		together to a complete design. Actually \Sc can be understood 
                as VHDL-AMS package as a way of grouping a collection of related 
                declarations that serve a common purpose.
                \smallskip
    	\end{minipage}\\ \hline
        \duc & \Clc &
	\begin{minipage}[c]{0.5\linewidth} 
                \smallskip
		An  subsistem which represent \duc.
    	\end{minipage}\\ \hline
	 \du & \Csc &
	\begin{minipage}[c]{0.5\linewidth}
	      \smallskip	 
              \Csc  describes \du. It can be identify with the primary unit (\ent) and secondary 
	      unit (\arch) structure in VHDL-AMS.
	      The \Csc's name corresponds to the name of the entity in VHDL-AMS.
              \smallskip
    	\end{minipage}\\ \hline
        \arch & \Ft & 
	\begin{minipage}[c]{0.5\linewidth} 
              \smallskip
	      The feature section of the \Csc, where the mathematical 
	      transfer function of the component is entered, holds the same information 
	      as the \arch section of the VHDL-AMS.
              \smallskip 
    	\end{minipage}\\ \hline

    \end{tabular}
\end{center}
