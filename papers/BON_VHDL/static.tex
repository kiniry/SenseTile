\begin{center}
\begin{tabular}{ c  c  c }
\multicolumn{3}{c}{\textbf{STATIC TYPES}}\\ 
\hline \hline
VHDL-AMS & BON & Description \\ 
\hline \hline
% first element
\begin{minipage}[c]{2.4cm}
\smallskip \smallskip
\centering
\emph{model}
\smallskip \smallskip
\end{minipage}
& 
\begin{minipage}[c]{4.6cm}
\smallskip \smallskip
\centering
\lstinlinen|SYSTEM_CHART|  
\smallskip \smallskip
\end{minipage}
& 
\begin{minipage}[c]{6cm}
         \smallskip \smallskip
	As an top-level description 
        (informal/formal), \lstinlinen|SYSTEM_CHART|
        connect all the model subsystems together 
        to a model.
        \smallskip \smallskip
\end{minipage}\\ 

% next element

\begin{minipage}[c]{2.4cm}
\smallskip \smallskip
\centering
model subsystem 
\smallskip \smallskip
\end{minipage}
& 
\begin{minipage}[c]{4.6cm}
\smallskip \smallskip
\centering
\lstinlinen|CLUSTER_CHART| 
\smallskip \smallskip
\end{minipage}
& 
\begin{minipage}[c]{6cm}
      \smallskip \smallskip	 
      An  subsistem which represent module colleaction.
        VHDL-AMS allows hierarchical structure modeling, 
	meaning that subsystems (modules), can be developed 
        independently.
      \smallskip \smallskip
\end{minipage}\\ 

% next element

\begin{minipage}[c]{2.4cm}
\smallskip \smallskip
\centering
modul
\smallskip \smallskip
\end{minipage}
& 
\begin{minipage}[c]{4.6cm}
\smallskip \smallskip
\centering
\lstinlinen|CLASS_CHART| 
\smallskip \smallskip
\end{minipage}
& 
\begin{minipage}[c]{6cm}
      \smallskip \smallskip	 
      \lstinlinen|CLASS_CHART|  describes module. 
      It can be identify with the primary unit (\ent) 
      and secondary unit (\arch) structure in VHDL-AMS. 
      The \lstinlinen|CLASS_CHART|'s name corresponds 
      to the name of the entity in VHDL-AMS.
      \smallskip \smallskip
\end{minipage}\\ 

% next element

\begin{minipage}[c]{2.4cm}
\smallskip \smallskip
\centering
\arch
\smallskip \smallskip
\end{minipage}
& 
\begin{minipage}[c]{4.6cm}
\smallskip \smallskip
\centering
\lstinlinen|feature|
\smallskip \smallskip
\end{minipage}
&
\begin{minipage}[c]{6cm} 
      \smallskip \smallskip
      The feature section of the \lstinlinen|CLASS_CHART|, 
      where the mathematical transfer function of the component 
      is entered, holds the same information as the \arch section 
      of the VHDL-AMS.
      \smallskip \smallskip 
\end{minipage}\\ 

\end{tabular}
\end{center}
