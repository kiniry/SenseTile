\begin{center}
\begin{tabular}{ c  c }
\multicolumn{2}{c}{}\\ 
INFORMAL BIT & FORMAL BIT \\ 
\begin{minipage}[l]{6.4cm} 
\begin{lstlisting}[language=Bon]
class_chart BIT
indexing
in_cluster: "TYPE";
created: "Dragan Stosic";
revised: "Joe Kiniry";
organisation: "UCD";
explanation
  "BIT model of VHDL-AMS bit."
query
  "What is your value ?",
command
  "Create a new initialized BIT.",
  "Set the bit value."
constraint
  "BIT's value are greater than or \
  \equal 0 and less than 2."
end
\end{lstlisting}

\end{minipage}  
&
\begin{minipage}[l]{6.6cm}
\begin{lstlisting}[language=Bon]
effective class BIT
  indexing
  about: "BIT model of VHDL-AMS bit.";
  title:        "BIT";
  author:       "Joe & Dragan";
  organisation: "UCD"; 
  inherit INTEGER
    feature{NONE}
      -- This BIT's value.
      bit : INTEGER 
        ensure
          Result /= Void;
        end
    feature
      -- Create a new initialized BIT.
      make : BIT -> aBit : INTEGER
        require   
\end{lstlisting}
\end{minipage}\\
\begin{minipage}[l]{6.4cm} 
The current IEEE standard VHDL-AMS language reference manual 
tries to define VHDL-AMS as well as possible in a descriptive way, 
explaining the semantics in English. As we have shown in example above,
BON has an easily-readable, English-based textual notation.
The two \emph{stepwise refinement} which was derived, 
from informal to formal specification, 
is represented using \emph{isomorphism} relation. 
This relation maps the following:
\begin{description}
\item[-] \lstinlinenb|query| $\rightsquigarrow$ \emph{getBit}.
\item[-] first \lstinlinenb|command| $\rightsquigarrow$ \lstinlinenb|make|.
\item[-] second \lstinlinenb|command| $\rightsquigarrow$ \emph{setBite}.
\item[-] \lstinlinenb|constraint| $\rightsquigarrow$ \lstinlinenb|invariant|.
\end{description}
\end{minipage}  
&
\begin{minipage}[l]{6.6cm}
\begin{lstlisting}[language=Bon]
          aBit >= 0 and aBit < 2;
        ensure
          delta bit;
          bit = old aBit;
        end
      -- What is your value?
      getBit : INTEGER 
        ensure
          Result /= Void;
        end
      --Set the bit value.
      setBit: Void -> aBit : INTEGER 
        require
          aBit >= 0 and aBit < 2;
        ensure
          delta bit;
          bit = old aBit;
        end
      --BIT's value are greater than or equal 0 
      --and less than 2.
      invariant
        bit >= 0 and bit < 2;
end    
\end{lstlisting}
\end{minipage}  
\end{tabular}
\end{center}
