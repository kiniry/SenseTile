%
% "Refining from the BON High-level Specification Language to the
% VHDL Hardware Description Language" for UNKNOWN09.
% Joseph R. Kiniry, Dragan Stosic
% $Id: paper.tex 2795 2007-09-23 19:35:53Z dmz $
%

\documentclass{article} 
\usepackage{times}

\usepackage{ifpdf}
\usepackage{a4wide}
\usepackage{pdfsync}

\ifpdf
\usepackage[pdftex]{graphicx}
\else
\usepackage{graphicx}
\fi

%\usepackage{todonotes}

% \usepackage{html}
% \usepackage{url}
\usepackage{xspace}
%\usepackage{doublespace}
\usepackage{tabularx}
\usepackage{epsfig}
\usepackage{amsmath}
\usepackage{amsfonts}
\usepackage{amssymb}
\usepackage{eucal}
\usepackage{stmaryrd}
% \ifpdf
% \usepackage[centredisplay]{diagrams}
% \else
% \usepackage[centredisplay,PostScript=dvips]{diagrams}
% \fi
\usepackage{float}

\ifpdf
\usepackage[pdftex,bookmarks=false,a4paper=false,
            plainpages=false,naturalnames=true,
            colorlinks=true,pdfstartview=FitV,
            linkcolor=blue,citecolor=blue,urlcolor=blue,
            pdfauthor="Joseph R. Kiniry and Dragan Stosic"]{hyperref}
\else
\usepackage[dvips,linkcolor=blue,citecolor=blue,urlcolor=blue]{hyperref}
\fi

\newcommand{\tablesize}{\footnotesize}
\newcommand{\eg}{e.g.,\xspace}
\newcommand{\ie}{i.e.,\xspace}
\newcommand{\etc}{etc.\xspace}
\newcommand{\myhref}[2]{\ifpdf\href{#1}{#2}\else\htmladdnormallinkfoot{#2}{#1}\fi}
%\newcommand{\myhref}[2]{\emph{#2}}
\newcommand{\todo}{\textbf{TODO:}}
\newcommand{\ST}{\emph{SenseTile}\xspace}
\newcommand{\STs}{\emph{SenseTiles}\xspace}
\newcommand{\STS}{\emph{SenseTile Simulator}\xspace}
\newcommand{\datastore}{\STs Scientific Datastore\xspace}
\newcommand{\computefarm}{The \STs Scientific Compute Farm\xspace}
\newcommand{\computefarmlong}{UCD CASL \STs Software and Data Compute Server Farm\xspace}
\newcommand{\sensorfarm}{The \STs\xspace}

%---------------------------------------------------------------------
% New commands, macros, \etc
%---------------------------------------------------------------------

%% \input{kt}

%=====================================================================

\begin{document}

\title{Refining from the BON High-level Specification Language to the
  VHDL Hardware Description Language}

\author{Joseph R. Kiniry and Dragan Stosic\\
UCD CASL: Complex and Adaptive Systems Laboratory and\\
School of Computer Science and Informatics,\\
University College Dublin,\\
Belfield, Dublin 4, Ireland,\\
kiniry@acm.org and dragan.stosic@gmail.com\\
}

\maketitle

%======================================================================
%\thispagestyle{empty}
\begin{abstract}
  
The \textbf{UCD CASL SenseTile System} is a large-scale, general-purpose
sensor system installed at the University College Dublin in Dublin,
Ireland.  \textbf{Our facility provides a capability---unique in the world in
terms of its scale and flexibility---for performing in-depth
investigation of both the specific and general issues in large-scale
sensor networking.}

This system integrates a sensor platform, a datastore, and a compute
farm.  The sensor platform is a custom-designed, but inexpensive,
sensor platform (called the \ST) paired with general-purpose
small-scale compute nodes, including everything from low-power PDAs to
powerful touchscreen-based portable computers.  The datastore is a
multi-terabyte scale scientific datastore into which sensor data
flows, and in which online and offline scientific computation of
sensor and other scientific data takes place.  The compute farm is a
large-scale Linux, Solaris, and OS~X-based compute farm used by
scientists for these online and offline scientific computations.

The \STs have several unusual built-in sensors and emitters, sensor
nodes can be extended via a general-purpose USB bus, and \STs contain
an FPGA that is dynamically updated on a per-experiment basis.  \STs
are custom-designed in VHDL.  Like many hardware-based systems, to
write new software for or against the \ST architecture one either uses
the actual hardware or one runs a \STS.  But this simulator emulates a
piece of hardware, one would like to know if the simulator
\emph{actually} simulates the hardware.  We have developed an approach
through which we refine the VHDL specification up to the formal BON
specification language, and from this specification we generate an
executable, functional model-based specification in the JML language.
We have used this specification to implement a fully validated and
verified \ST simulator.

\end{abstract}

%======================================================================
\section{Introduction}

An introduction.

%=====================================================================
\section{Conclusion}

A conclusion.

%======================================================================
%% \nocite{ex1,ex2}
\bibliographystyle{plain}
\bibliography{extra,%
              abbrev,%
              ads,%
              category,%
              complexity,%
              hypertext,%
              icsr,%
              knowledge,%
              languages,%
              linguistics,%
              meta,%
              metrics,%
              misc,%
              modeling,%
              modeltheory,%
              reuse,%
              rewriting,%
              softeng,%
              specification,%
              ssr,%
              technology,%
              theory,%
              web,%
              upcoming,%
              upcoming_conferences,%
              conferences,%
              workshops,%
              verification,%
              escjava,%
              jml,%
              nijmegen}

%======================================================================
% Fin

\end{document}

%%% Local Variables: 
%%% mode: latex
%%% eval: 
%%% TeX-master: t
%%% End: 
