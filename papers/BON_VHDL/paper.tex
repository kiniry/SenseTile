%
% "Refining from the BON High-level Specification Language to the
% VHDL Hardware Description Language" for UNKNOWN09.
% Joseph R. Kiniry, Dragan Stosic
% $Id: paper.tex 2795 2007-09-23 19:35:53Z dmz $
%

\documentclass{article} 
\usepackage{times}
\usepackage{multirow}
\usepackage{ifpdf}
\usepackage{a4wide}
\usepackage{pdfsync}
\usepackage{amsmath}
\usepackage{listings}

\ifpdf
\usepackage[pdftex]{graphicx}
\else
\usepackage{graphicx}
\fi

%\usepackage{todonotes}

% \usepackage{html}
% \usepackage{url}
\usepackage{xspace}
%\usepackage{doublespace}
\usepackage{tabularx}
\usepackage{epsfig}
\usepackage{amsmath}
\usepackage{amsfonts}
\usepackage{amssymb}
\usepackage{eucal}
\usepackage{stmaryrd}
% \ifpdf
% \usepackage[centredisplay]{diagrams}
% \else
% \usepackage[centredisplay,PostScript=dvips]{diagrams}
% \fi
\usepackage{float}

\ifpdf
\usepackage[pdftex,bookmarks=false,a4paper=false,
            plainpages=false,naturalnames=true,
            colorlinks=true,pdfstartview=FitV,
            linkcolor=blue,citecolor=blue,urlcolor=blue,
            pdfauthor="Joseph R. Kiniry and Dragan Stosic"]{hyperref}
\else
\usepackage[dvips,linkcolor=blue,citecolor=blue,urlcolor=blue]{hyperref}
\fi


\lstset{
  language=Java,
  basicstyle=\footnotesize\sffamily,
  identifierstyle=\itshape,
  stringstyle=\footnotesize\ttfamily,
  commentstyle=\textup,
  columns=flexible,
  captionpos=b,
  morekeywords={},
  literate=
  {-}{{$\textendash$}}1
  {<}{{$\textless$}}1
  {<=}{{$\textless=$}}1
  {>}{{$\textgreater$}}2
  {>=}{{$\textgreater=$}}2
  {\\}{{$\backslash\backslash$}}1
  {'}{{$\textprimstress$}}1
}

\def\lstinlinen{\lstinline[language=Bon,basicstyle=\normalsize\upshape]}

\def\lstinlinenv{\lstinline[language=VHDL,basicstyle=\normalsize\upshape]}

\def\lstinlinenc{\lstinline[language=VHDL,basicstyle=\normalsize\bfseries]}

\def\lstinlinenb{\lstinline[language=Bon,basicstyle=\normalsize\bfseries]}

\lstdefinelanguage{Bon} {
  morekeywords={require,ensure,feature,old,and,o,Q'slew,end,Result, Void},
  morekeywords={class_chart,class,delta,invariant,inherit,deferred,effective,NONE,make},
  morekeywords={explanation,query,command,constraint},
  morecomment=[l]{--}, 
  morecomment=[s]{/*}{*/}, 
  morestring=[b]", 
 literate=,
  showstringspaces=false
}

\lstdefinelanguage{VHDL} {
  morekeywords={not,xnor,and,nand, nor,or,xor,integer,byte,real},
  morekeywords={character,enum,type,boolean,bit,time,null,std_ulogic},
  morekeywords={sll, srl, sla, sra, rol, ror},
  morekeywords={rem,abs,mod},
  morekeywords={terminal,T'reference, T'contribution,T'tolerance},
  morekeywords={reference,contribution},
  morekeywords={nature,subnature,electrical,magnetic,translational,translational_v},
  morekeywords={rotational,rotational_v,fluidic,thermal,radiant},
  morekeywords={voltage,angular_velocity,pressure,velocity},
  morekeywords={current,force,torque,volumetric,flow_rate},
  morekeywords={T'left,T'right,T'high,T'ascending,T'image(x),T'value(s)},
  morekeywords={left,right,high,ascending,image,value},
  morekeywords={T'pos(s),T'val(x),T'succ(x),T'pred(x)},
  morekeywords={pos,val,succ,pred},
  morekeywords={A'left(n),A'right(n),A'low(n),A'high(n)},
  morekeywords={left,right,low,high},
  morekeywords={A'range(n),A'reverse_range(n),A'length(n),A'ascending(n)},
  morekeywords={range,reverse_range,length,ascending},
  morekeywords={T'leftof(x),T'rightof(x),T'base},
  morekeywords={leftof,rightof,base},
  morekeywords={Q'tolerance, Q'above,Q'delayed,Q'dot,Q'integ},
  morekeywords={delayed,dot,integ},
  morekeywords={Q'slew,Q'ztf,Q'ltf,Q'zoh},
  morekeywords={N'across, N'through},
  morekeywords={across,through,zoh,ltf,slew,ztf},
  morekeywords={above,tolerance,group,quantity},
  morecomment=[l]{--}, 
  morecomment=[s]{/*}{*/}, 
  morestring=[b]", 
 literate=,
  showstringspaces=false
}

\newcommand{\tablesize}{\footnotesize}
\newcommand{\eg}{e.g.,\xspace}
\newcommand{\ie}{i.e.,\xspace}
\newcommand{\etc}{etc.\xspace}
\newcommand{\myhref}[2]{\ifpdf\href{#1}{#2}\else\htmladdnormallinkfoot{#2}{#1}\fi}
%\newcommand{\myhref}[2]{\emph{#2}}
\newcommand{\todo}{\textbf{TODO:}}
\newcommand{\ST}{\emph{SenseTile}\xspace}
\newcommand{\STs}{\emph{SenseTiles}\xspace}
\newcommand{\STS}{\emph{SenseTile Simulator}\xspace}
\newcommand{\datastore}{\STs Scientific Datastore\xspace}
\newcommand{\computefarm}{The \STs Scientific Compute Farm\xspace}
\newcommand{\computefarmlong}{UCD CASL \STs Software and Data Compute Server Farm\xspace}
\newcommand{\sensorfarm}{The \STs\xspace}
\newcommand{\ent}{\emph{entity}\xspace}
\newcommand{\arch}{\emph{architecture}\xspace}
\newcommand{\archs}{\emph{architectures}\xspace}

\newcommand{\undef}{\emph{undefined}\xspace}
\newcommand{\arcoss}{\emph{across}\xspace}
\newcommand{\through}{\emph{through}\xspace}
%---------------------------------------------------------------------
% New commands, macros, \etc
%---------------------------------------------------------------------

%% \input{kt}

%=====================================================================

\begin{document}

\title{Refining from the BON High-level Specification Language to the
  VHDL-AMS Hardware Description Language}

\author{Joseph R. Kiniry and Dragan Stosic\\
UCD CASL: Complex and Adaptive Systems Laboratory and\\
School of Computer Science and Informatics,\\
University College Dublin,\\
Belfield, Dublin 4, Ireland,\\
kiniry@acm.org and dragan.stosic@gmail.com\\
}

\maketitle

%======================================================================
%\thispagestyle{empty}
\begin{abstract}
  
The \textbf{UCD CASL SenseTile System} is a large-scale, general-purpose
sensor system installed at the University College Dublin in Dublin,
Ireland.  \textbf{Our facility provides a capability---unique in the world in
terms of its scale and flexibility---for performing in-depth
investigation of both the specific and general issues in large-scale
sensor networking.}

This system integrates a sensor platform, a datastore, and a compute
farm.  The sensor platform is a custom-designed, but inexpensive,
sensor platform (called the \ST) paired with general-purpose
small-scale compute nodes, including everything from low-power PDAs to
powerful touchscreen-based portable computers.  The datastore is a
multi-terabyte scale scientific datastore into which sensor data
flows, and in which online and offline scientific computation of
sensor and other scientific data takes place.  The compute farm is a
large-scale Linux, Solaris, and OS~X-based compute farm used by
scientists for these online and offline scientific computations.

The \STs, as an embedded system, have several unusual built-in sensors 
and emitters, sensor nodes can be extended via a general-purpose USB bus, 
and \STs contain an FPGA that is dynamically updated on a per-experiment 
basis.  \STs are custom-designed in VHDL and VHDL-AMS.  Like many hardware-based systems,
 to write new software for or against the \ST architecture one either uses
the actual hardware or one runs a \STS.  But this simulator emulates a
piece of hardware, one would like to know if the simulator
\emph{actually} simulates the hardware.  We have developed an approach
through which we refine the VHDL specification up to the formal BON
specification language, and from this specification we generate an
executable, functional model-based specification in the JML language.
We have used this specification to implement a fully validated and
verified \ST simulator.

\end{abstract}

%======================================================================
\section{Introduction}
The \STs, represents heterogeneous collection of digital, analog 
and mixed signal hardware components. Embedded system complexity 
requires the need to strive for formal descriptions each component 
separatelly in order to produce specification of the \STs simulator. 
The development of complex system requires sophisticated tool and methodology 
in order to be able to complete a correct design in time.  
The authors will illustrate in this paper how BON can represent a high 
level analogue mapping and will further illustrate a procedure for 
transformation from VHDL-AMS to BON. As an extremly simple example, 
we will give practical aspect of refinement from VHDL-AMS to BON,
applied on A/D convertor, in order to produce formal specification 
and software simulation. A comprehensive list of the mapping between BON 
and VHDL-AMS can be found in the appendix.
%=====================================================================


\section{VHDL-AMS}
IEEE VHDL-AMS 1076.1 is the industry standard mixed-signal 
high-level description language for electronic and multi-domain 
systems. VHDL-AMS, as an superset of VHDL, supports modeling and 
simulation of analog and mixed-signal systems. The main contribution 
of VHDL-AMS is ability to describe the behavior of complex system 
trought differential and algebraic equations (DAEs). For this purpose, 
VHDL-AMS introduces the following capabilities and therms:

\begin{itemize}
\item 
\emph{Terminals} represents phisical connection points or circuit nodes 
of a network connectivity. They are declared to be of various 
\emph{natures}, which represent different energy domains of a system.
Terminals hold no values; their only role is to facilitate the formation
of conservative equation sets that, in turn, constrain the associated 
\emph{branch quatities}. 
\item
The \emph{simultaneous statements} creates one or more characteristic 
equations that are solved by the dedicated simulation kernel a.k.a.
"analog solver", to model continuously valued behaviour.  
\item
The \emph{quantities} denote a waveform or a time series of values.
The behaviour of the \emph{quantities} is given by the \emph{simultaneous
 statements}. They take values as a result of solving the set of simultaneous 
ordinary differential and algebraic equations (DAEs). To simplify, 
quantities can be understood as analog unknowns of (DAEs) that specify 
analog behavior. There are three kinds of quantity in VHDL-AMS: free, 
branch and source quantities. A \emph{free quantity} can be used in 
signal-flow modeling. A \emph{branch quantity} is used to model conservative energy
systems. A \emph{source quantity} is used for frequency and noise modeling.
\item
The \emph{break statement} is an construct which allows to 
initialize the DAE system at the beginning of a transient 
simulation and to mark discontinuities during such a transient 
simulation. A break statement allows restart of the analog kernel,
than overrides quantity values and executes concurrently with the analogue model. 
For any concurrent break statement there is an associated process.
\end{itemize} 
 
\section{Choice of BON as the modeling tool}
%Brief description and motivation.
The Business Object Notation (BON) is an analysis and design notation 
for object-oriented systems  which emphasizes \emph{seamlessness}, 
\emph{reversibility},software \emph{contracting} and \emph{refinement}. 
Let us define these principles precisely and point out why the principles 
are important in VHDL-AMS simulation.

\subsection{Seamlessness and Reversibility}
\emph{Seamlessness} is the use of a continuous process 
throughout the software lifecycle.
\emph{Reversibility} is the support for both forward and 
backward development process: from analysis to design 
and implementation, and back.
In VHDL-AMS, modelling represents an continious top-down-bottom-up
process. Top-down approach serves for design while bottom-up
serves for verification. The possibility of seamlessly translating 
changes made during a certain development phase back into earlier phases,
facilitates a stable and predictable environment and ensures the various 
models consistent with the code. Keeping in sync various models, documetation
and specification from forward progress to backward and redesign, preclude 
that design and implementation diverge to an inconsistent state. 
\emph{Seamlessness} also plays important role in relating behaviours at 
differing levels of abstraction. For example, one may wish to relate the 
structural views of an A/D converter at the level of individual bits and 
at the level of busses. Therefore capturing similarities between the 
differing levels of abstractions far outweigh their inevitable differences, 
making it possible to obtain a continuous process that ensures a direct 
mapping between a formal specification and its software solution.
    
\subsection{Contracting and Refinement}
\emph{Contracting} is the definition, for each software element, 
of the associated semantic properties. BON use a full assertion mechanism, 
allowing analysts to specify not only the structure of a system but also its
semantics ( constraints and invariants of the expected results ). 
In VHDL-AMS \emph{simultaneous statement} represents analogue constraint of 
continuous-time waveform. In other words, analog kernel calculate quantities such 
that the constraints specified by the simultaneous statements are satisfied. 
We will agree that the constraint of waveform is nothing else than \emph{invariant}.
Another example of \emph{invariant} is conservation of energy law in which 
the summation of \arcoss quantities (\texttt{$\sum_{loop}$ across quatities =0}) 
around any loop equals zero and the summation of through 
quantities (\texttt{$\sum_{node}$ through quatities =0}) at any node equals zero. 
Moreover, any predefined (implicite) attribute in VHDL 
has defined pre(post) condition trought input(output) 
parameters and theirs constraints. Furthermore any subtype(subnature) 
represents restricted set of values from a base type(nature). 
The condition that determines which values are in the subtype(subnature)  
may be contracted by specifying characteristic predicate. 
For example, in typed $\lambda$ calculi, the VHDL-AMS subtype \lstinlinenc|bit| of 
\lstinlinenc|integer|, may be defined as \texttt{bit:=$\lambda$ n:nat.n < 2}. 
The corresponding informal and formal BON transformations are shown 
by the example below:
\begin{center}
\begin{tabular}{ c  c }
\multicolumn{2}{c}{}\\ 
INFORMAL BIT & FORMAL BIT \\ 
\begin{minipage}[l]{6.4cm} 
\begin{lstlisting}[language=Bon]
class_chart BIT
indexing
in_cluster: "TYPE";
created: "Dragan Stosic";
revised: "Joe Kiniry";
organisation: "UCD";
explanation
  "BIT model of VHDL-AMS bit."
query
  "What is your value ?",
command
  "Create a new initialized BIT.",
  "Set the bit value."
constraint
  "BIT's value are greater than or \
  \equal 0 and less than 2."
end
\end{lstlisting}

\end{minipage}  
&
\begin{minipage}[l]{6.6cm}
\begin{lstlisting}[language=Bon]
effective class BIT
  indexing
  about: "BIT model of VHDL-AMS bit.";
  title:        "BIT";
  author:       "Joe & Dragan";
  organisation: "UCD"; 
  inherit INTEGER
    feature{NONE}
      -- This BIT's value.
      bit : INTEGER 
        ensure
          Result /= Void;
        end
    feature
      -- Create a new initialized BIT.
      make : BIT -> aBit : INTEGER
        require   
\end{lstlisting}
\end{minipage}\\
\begin{minipage}[l]{6.4cm}
The construction technique which was implemented in example above is
\emph{refinement calculus}. 
This concept  can be understud as series of correctness-preserving 
transformations from general perspective to a precise level of detail.\\
The \emph{refinement calculus} which was derived in two steps, 
from informal to formal specification, is represented using relation 
$\rightsquigarrow$, which is called \emph{refinement}.
This relation actually represents an \emph{isomorphism} and maps the following:
\begin{description}
\item[-] \lstinlinenb|query| $\rightsquigarrow$ \emph{getBit}.
\item[-] first \lstinlinenb|command| $\rightsquigarrow$ \lstinlinenb|make|.
\item[-] second \lstinlinenb|command| $\rightsquigarrow$ \emph{setBit}.
\item[-] \lstinlinenb|constraint| $\rightsquigarrow$ \lstinlinenb|invariant|.
\end{description}
\end{minipage}  
&
\begin{minipage}[l]{6.6cm}
\begin{lstlisting}[language=Bon]
          aBit >= 0 and aBit < 2;
        ensure
          delta bit;
          bit = old aBit;
        end
      -- What is your value?
      getBit : INTEGER 
        ensure
          Result /= Void;
        end
      --Set the bit value.
      setBit: Void -> aBit : INTEGER 
        require
          aBit >= 0 and aBit < 2;
        ensure
          delta bit;
          bit = old aBit;
        end
      --BIT's value are greater than or equal 0 
      --and less than 2.
      invariant
        bit >= 0 and bit < 2;
end    
\end{lstlisting}
\end{minipage}  
\end{tabular}
\end{center}

Another advantage of BON as mapping tool is an easily-readable, 
English-based textual notation. The current IEEE standard VHDL-AMS 
language reference manual tries to define VHDL-AMS as well as possible 
in a descriptive way, explaining the semantics in English.
Therefore as an notition based on contract design, and as  methodology of 
best modelling practices, BON is the obvious choice for the our simulator.
Throughout this paper we will try to make this more clearly.  

\section{Mapping between VHDL-AMS and BON}


\section{Refinement from VHDL-AMS to BON }
 
An VHDL-AMS design units are described using an \ent-\arch structure. 
The design unit contains an \ent and one or more \archs. 
The \ent specifies the interface that an entity provides of itself 
to the outside. In other words, \ent introduces a name, list of 
input/output ports and the definition of the generic parameters. 
The textual presentation of \ent is shown in Figure.1. 
{\small
\begin{tabbing}
\ \ \ \= \textbf{library} ieee\_proposed; \textbf{use} ieee\_proposed.electrical\_systems.\textbf{all}; \\ 
\ \ \ \= \textbf{entity} adc \textbf{is} \\
\> \ \ \ \ \ \ \ \ \= \textbf{port} ( \textbf{quantity} gain : \textbf{in} voltage; \\
\> \> \ \ \ \ \ \ \ \ \ \ \ \= \textbf{terminal} a : electrical; \\
\> \> \ \ \ \ \ \ \ \ \ \ \ \= \textbf{signal} clk : \textbf{in} bit; \\
\> \> \ \ \ \ \ \ \ \ \ \ \ \= \textbf{signal} d\_out : \textbf{out} bit ); \\
\ \ \ \= \textbf{end entity} adc;
\ \ \ \     \\
\ \ \ \     \\
\ \ \ \ Figure 1. An \ent body for A/D converter.
\end{tabbing}
%\vspace{-18pt}
}


 
The \arch body represents description of the internal implementation 
of an \ent including only process statements(encapsulation), which 
are collections  of actions to be executed in sequence, or simultaneous statements, 
which describe equations of analogue system behaviour. 
The textual presentation of \arch is shown in Figure.2.
{\small
\begin{tabbing}
\ \ \ \= \textbf{architecture} simple \textbf{of} adc \textbf{is} \\
\> \ \ \ \ \ \ \ \ \ \ \ \= \textbf{constant} ref : real := 5.0; \\
\> \ \ \ \ \ \ \ \ \ \ \ \= \textbf{quantity} v\_in \textbf{across} a; \\
\> \ \ \ \ \ \ \ \ \ \ \ \= \textbf{quantity} v\_amplified : voltage; \\
\ \ \ \= \textbf{begin} \\
\> \ \ \ \ \ \ \ \ \ \ \ \= v\_amplified == v\_in * gain; \\
\> \ \ \ \ \ \ \ \ \ \ \ \= adc\_behavior:\textbf{process is} \\
\> \ \ \ \ \ \ \ \ \ \ \ \= \textbf{variable} stored\_d : bit; \\
\> \ \ \ \ \ \ \ \ \ \ \ \= \textbf{begin} \\
\> \> \ \ \ \ \ \ \ \ \ \ \= \textbf{if} clk = '1' \textbf{then} \\
\> \> \> \ \ \ \ \ \ \ \ \ \= \textbf{if} v\_amplified $>$ ref / 2.0 \textbf{then} \\
\> \> \> \> \ \ \ \ \ \ \ \ \= stored\_d := '1'; \\
\> \> \> \ \ \ \ \ \ \ \ \ \= \textbf{else} \\
\> \> \> \> \ \ \ \ \ \ \ \ \= stored\_d := '0'; \\
\> \> \> \ \ \ \ \ \ \ \ \ \= \textbf{end if}; \\
\> \> \ \ \ \ \ \ \ \ \ \ \=\textbf{end if};\\
\> \> \ \ \ \ \ \ \ \ \ \ \= d\_out $<$= stored\_d \textbf{after} 5 ns;\\ \\
\> \> \ \ \ \ \ \ \ \ \ \ \= \textbf{wait on} clk;\\
\> \ \ \ \ \ \ \ \ \ \ \ \= \textbf{end process} adc\_behavior;\\
\ \ \ \= \textbf{end architecture} simple;
\ \ \ \     \\
\ \ \ \     \\
\ \ \ \ Figure 2. A behavioral \arch body for A/D converter.
\end{tabbing}
}







%=====================================================================
\section{Conclusion}

A conclusion.

\section {Acknowledgments}
The research upon which this paper is based was caried out with
sponsorship from Science Foundation Ireland via the 
\textbf{UCD CASL SenseTile System} grant.


%% \nocite{ex1,ex2}
\bibliographystyle{plain}
\bibliography{extra,%
              abbrev,%
              ads,%
              category,%
              complexity,%
              hypertext,%
              icsr,%
              knowledge,%
              languages,%
              linguistics,%
              meta,%
              metrics,%
              misc,%
              modeling,%
              modeltheory,%
              reuse,%
              rewriting,%
              softeng,%
              specification,%
              ssr,%
              technology,%
              theory,%
              web,%
              upcoming,%
              upcoming_conferences,%
              conferences,%
              workshops,%
              verification,%
              escjava,%
              jml,%
              nijmegen}

%======================================================================
% Fin

%======================================================================
\appendix
\newpage
\section {APPENDIX}
\begin{center}
    \begin{tabular}{|c|c|l|}
        \hline
        \multicolumn{3}{|c|}{Static mapping}\\ \hline
    	\hline
    	VHDL-AMS & BON & Description \\ \hline
    	\ducs & \Sc & 
	\begin{minipage}[c]{0.5\linewidth}
                 \smallskip
    		VHDL-AMS allows hierarchical structure modeling, 
    		meaning that subsistems (\ducs), can be developed independently.
    		As an top-level description (informal/formal), \Sc connect all the \ducs
    		together to a complete design. Actually \Sc can be understood 
                as VHDL-AMS package as a way of grouping a collection of related 
                declarations that serve a common purpose.
                \smallskip
    	\end{minipage}\\ \hline
        \duc & \Clc &
	\begin{minipage}[c]{0.5\linewidth} 
                \smallskip
		An  subsistem which represent \duc.
    	\end{minipage}\\ \hline
	 \du & \Csc &
	\begin{minipage}[c]{0.5\linewidth}
	      \smallskip	 
              \Csc  describes \du. It can be identify with the primary unit (\ent) and secondary 
	      unit (\arch) structure in VHDL-AMS.
	      The \Csc's name corresponds to the name of the entity in VHDL-AMS.
              \smallskip
    	\end{minipage}\\ \hline
        \arch & \Ft & 
	\begin{minipage}[c]{0.5\linewidth} 
              \smallskip
	      The feature section of the \Csc, where the mathematical 
	      transfer function of the component is entered, holds the same information 
	      as the \arch section of the VHDL-AMS.
              \smallskip 
    	\end{minipage}\\ \hline

    \end{tabular}
\end{center}

\begin{center}
\begin{tabular}{ c  c  c }
\multicolumn{3}{c}{\textbf{ARITHMETIC OPERATORS}}\\ 
\hline \hline
VHDL-AMS & BON & Description \\ 
\hline \hline 
\begin{minipage}[c]{2.4cm} 
 \centering
 \smallskip \smallskip 
\lstinlinenc|**|
 \smallskip \smallskip
\end{minipage}  
        &
\begin{minipage}[c]{4.6cm}
\centering
\smallskip \smallskip 
\textbf{\textasciicircum}
\smallskip \smallskip
\end{minipage}  
        & 
\begin{minipage}[c]{6cm}  
\smallskip \smallskip
exponentiation.
\smallskip \smallskip 
\end{minipage}\\ 

\begin{minipage}[c]{2.4cm} 
 \centering
 \smallskip \smallskip 
 \lstinlinenc|+,-, *, /|
 \smallskip \smallskip
\end{minipage}  
        &
\begin{minipage}[c]{4.6cm}
 \centering
 \smallskip \smallskip 
 \lstinlinenc|+,-, *, /|
 \smallskip \smallskip
\end{minipage}  
        & 
\begin{minipage}[c]{6cm}  
 \smallskip \smallskip
 identity, addition, negation, subtraction, multiplication, division.
 \smallskip \smallskip        
\end{minipage}\\ 
\begin{minipage}[c]{2.4cm} 
 \centering
\smallskip \smallskip 
 \lstinlinenc|rem|
\smallskip \smallskip
\end{minipage}  
        &
\begin{minipage}[c]{4.6cm}
 \centering
\smallskip \smallskip 
\undef
\smallskip \smallskip
\end{minipage}  
        & 
\begin{minipage}[c]{6cm}  
\smallskip \smallskip
reminder.
\smallskip \smallskip
\end{minipage}\\ 


\begin{minipage}[c]{2.4cm} 
\centering
\smallskip \smallskip 
\lstinlinenc|abs|
\smallskip \smallskip
\end{minipage}  
&
\begin{minipage}[c]{4.6cm}
\centering
\smallskip \smallskip 
\undef
\smallskip \smallskip
\end{minipage}  
& 
\begin{minipage}[c]{6cm}  
\smallskip \smallskip
absolute value.
\smallskip \smallskip       
\end{minipage}\\ 

\begin{minipage}[c]{2.4cm} 
\centering 
\smallskip \smallskip
\lstinlinenc|mod|
\smallskip \smallskip
\end{minipage}  
        &
\begin{minipage}[c]{4.6cm}
\centering
\smallskip \smallskip 
\lstinlinenb|\\|
\smallskip \smallskip
\end{minipage}  
& 
\begin{minipage}[c]{6cm}  
\smallskip \smallskip
 modulo.
\smallskip \smallskip
\end{minipage} \\
\end{tabular}
\end{center}

\begin{center}
\begin{tabular}{ c  c  c }
\multicolumn{3}{c}{\textbf{SHIFT OPERATORS}}\\ 
\hline \hline
VHDL-AMS & BON & Description \\ 
\hline \hline

\begin{minipage}[c]{2.4cm} 
\centering
\smallskip \smallskip 
\lstinlinenc|sll, srl, sla| \\
\lstinlinenc|sra, rol, ror|
\smallskip \smallskip
\end{minipage}  
&
\begin{minipage}[c]{4.6cm}
\centering 
\smallskip \smallskip
\lstinlinen|undefined|
\smallskip \smallskip
\end{minipage}  
& 
\begin{minipage}[c]{6cm}  
\smallskip \smallskip
shift-left/right logical,\\
shift-left/right arithmetic,\\
rotate left/right.
\smallskip \smallskip
\end{minipage}\\ 
\end{tabular}
\end{center}

\begin{center}
\begin{tabular}{ c  c  c }
\multicolumn{3}{c}{\textbf{RELATIONAL OPERATORS}}\\ 
\hline \hline
VHDL-AMS & BON & Description \\ 
\hline \hline


\begin{minipage}[c]{2.4cm} 
\centering
\smallskip \smallskip 
\lstinlinenc|=, /=, <, >| \\
\lstinlinenc|<=, >=| 
\smallskip \smallskip
\end{minipage}  
&
\begin{minipage}[c]{4.6cm}
\centering 
\smallskip \smallskip
\lstinlinenb|=, /=, <, >| \\
\lstinlinenb|<=, >=|
\smallskip \smallskip
\end{minipage}  
& 
\begin{minipage}[c]{6cm} 
\smallskip \smallskip
equality, inequality, less than, greater than
 less than or equal , greater than or equal.
\smallskip \smallskip
\end{minipage}\\
\end{tabular}
\end{center}

\begin{center}
\begin{tabular}{ c  c  c }
\multicolumn{3}{c}{\textbf{LOGICAL OPERATORS}}\\ 
\hline \hline
VHDL-AMS & BON & Description \\ 
\hline \hline 


\begin{minipage}[c]{2.4cm} 
\centering
\smallskip \smallskip 
\lstinlinenc|not, and | \\
\lstinlinenc|or, xor|
\smallskip \smallskip
\end{minipage}  
&
\begin{minipage}[c]{4.6cm}
\centering
\smallskip \smallskip 
\lstinlinenc|not, and | \\
\lstinlinenc|or, xor|
\smallskip \smallskip
\end{minipage}  
& 
\begin{minipage}[c]{6cm} 
\smallskip \smallskip
negation, logical and, \\
logical or, exclusive or.
\smallskip \smallskip 
\end{minipage}\\ 


\begin{minipage}[c]{2.4cm} 
\centering 
\smallskip \smallskip
\lstinlinenc|nand|
\smallskip \smallskip
\end{minipage}  
&
\begin{minipage}[c]{4.6cm}
\centering 
\smallskip \smallskip
\lstinlinenc|not and|
\smallskip \smallskip
\end{minipage}  
& 
\begin{minipage}[c]{6cm} 
\smallskip \smallskip
negated logical and.
\smallskip \smallskip
\end{minipage}\\ 
	
\begin{minipage}[c]{2.4cm} 
\centering 
\smallskip \smallskip
\lstinlinenc|nor|
\smallskip \smallskip
\end{minipage}  
&
\begin{minipage}[c]{4.6cm}
\centering 
\smallskip \smallskip
\lstinlinenc|not or|
\smallskip \smallskip
\end{minipage}  
& 
\begin{minipage}[c]{6cm} 
\smallskip \smallskip
negated logical or.
\smallskip \smallskip
\end{minipage}\\ 

\begin{minipage}[c]{2.4cm} 
\centering 
\smallskip \smallskip
\lstinlinenc|xnor|
\smallskip \smallskip
\end{minipage}  
&
\begin{minipage}[c]{4.6cm}
\centering 
\smallskip \smallskip
\lstinlinenc|not xor|
\smallskip \smallskip
\end{minipage}  
& 
\begin{minipage}[c]{6cm}  
\smallskip \smallskip
negated exclusive or.
\smallskip \smallskip
\end{minipage}\\ 
\end{tabular}
\end{center}

\begin{center}
    \begin{tabular}{|c|c|l|}
        \hline
        \multicolumn{3}{|c|}{\textbf{SCALAR TYPES/SUBTYPES}}\\ \hline
    	\hline
    	VHDL-AMS & BON & Description \\ \hline
    	\begin{minipage}[c]{3cm} 
         \centering 
         \textbf{integer},\textbf{byte},\\ 
         \emph{natural},\\
         \emph{positive}
        \end{minipage}  
        &
       \begin{minipage}[c]{3cm}
         \centering 
         \emph{INTEGER}
        \end{minipage}  
        & 
	\begin{minipage}[c]{0.5\linewidth}  
                 \vskip 1mm
    		int, byte,\\
                integer  \textbf{range} 0 \textbf{to} \emph{highest\_integer},\\
                integer  \textbf{range} range 1 \textbf{to} \emph{highest\_integer}.
                \smallskip
    	\end{minipage}\\ \hline

       \begin{minipage}[c]{3cm} 
         \centering 
         \textbf{real}
        \end{minipage}  
        &
       \begin{minipage}[c]{3cm}
         \centering 
         \emph{REAL}
        \end{minipage}  
        & 
	\begin{minipage}[c]{0.5\linewidth}  
                 \vskip 1mm
    		Floating-point.
    	\end{minipage}\\ \hline


       \begin{minipage}[c]{3cm} 
         \centering 
         \textbf{character}
        \end{minipage}  
        &
       \begin{minipage}[c]{3cm}
         \centering 
         \emph{CHARACTER}
        \end{minipage}  
        & 
	\begin{minipage}[c]{0.5\linewidth} 
                 \vskip 1mm
    		Character.
    	\end{minipage}\\ \hline
      
      \begin{minipage}[c]{3cm} 
         \centering 
         \textbf{enumeration type}
        \end{minipage}  
        &

       \begin{minipage}[c]{3cm}
         \centering 
         \emph{SEQUENCE[TYPE]}
        \end{minipage}  
        & 
	\begin{minipage}[c]{0.5\linewidth} 
                 \vskip 1mm
    		Type sequence.
    	\end{minipage}\\ \hline


    \begin{minipage}[c]{3cm} 
         \centering 
         \textbf{boolean}
        \end{minipage}  
        &

       \begin{minipage}[c]{3cm}
         \centering 
         \emph{BOOLEAN}
        \end{minipage}  
        & 
	\begin{minipage}[c]{0.5\linewidth} 
                 \vskip 1mm
    		True, False.
    	\end{minipage}\\ \hline

       \begin{minipage}[c]{3cm} 
         \centering 
         \textbf{bit}
        \end{minipage}  
        &

       \begin{minipage}[c]{3cm}
         \centering 
         \emph{BOOLEAN}
        \end{minipage}  
        & 
	\begin{minipage}[c]{0.5\linewidth} 
                 \vskip 1mm
    		Predefined enumeration type. The logical operations for Boolean values
                can also be applied to values of type \textbf{bit}. Therefore Values \textbraceleft'0','1'\textbraceright  corresponds to \emph{BOOELAN} false/true.
                \smallskip
    	\end{minipage}\\ \hline
      
       \begin{minipage}[c]{3cm} 
         \centering 
         \textbf{time}
        \end{minipage}  
        &
       \begin{minipage}[c]{3cm}
         \centering 
         \emph{undefined}
        \end{minipage}  
        & 
	\begin{minipage}[c]{0.5\linewidth} 
                 \vskip 1mm
    		Predefined physical type. The time is used extensively to specify delays.
                \smallskip
    	\end{minipage}\\ \hline \hline
        \multicolumn{3}{|c|}{\textbf{Digitally encoded values mapping}}\\ \hline \hline
        \begin{minipage}[c]{3cm} 
         \centering 
         \textbf{std\_ulogic}
        \end{minipage}  
        &
       \begin{minipage}[c]{3cm}
         \centering 
         \emph{SEQUENCE}\\\emph{[CHARACTER]}
        \end{minipage}  
        & 
	\begin{minipage}[c]{0.5\linewidth} 
                 \vskip 1mm
    		Predefined enumeration type.\\
                        U', Uninitialized \\
                       'X', Forcing Unknown \\
                       '0', Forcing 0 \\
                       '1', Forcing 1 \\
                       'Z', High Impedance \\
                       'W', Weak Unknown \\
                       'L', Weak 0 \\
                       'H', Weak 1 \\
                       '-'  Don't care
        \smallskip
    	\end{minipage}\\ \hline

    \end{tabular}
\end{center}

\begin{center}
\begin{tabular}{ c  c  c }
\multicolumn{3}{c}{\textbf{VALUE AND SIGNAL ASSIGNMENT}}\\ 
\hline \hline
VHDL-AMS & BON & Description \\ 
\hline \hline


\begin{minipage}[c]{2.4cm} 
\centering
\smallskip \smallskip 
\lstinlinenc|:=| 
\smallskip \smallskip
\end{minipage}  
&
\begin{minipage}[c]{4.6cm}
\centering 
\smallskip \smallskip
\lstinlinenb|:=|
\smallskip \smallskip
\end{minipage}  
& 
\begin{minipage}[c]{6cm} 
\smallskip \smallskip
Value assignment.
\smallskip \smallskip
\end{minipage}\\


\begin{minipage}[c]{2.4cm} 
\centering
\smallskip \smallskip 
\lstinlinenc|<=| 
\smallskip \smallskip
\end{minipage}  
&
\begin{minipage}[c]{4.6cm}
\centering 
\smallskip \smallskip
\undef
\smallskip \smallskip
\end{minipage}  
& 
\begin{minipage}[c]{6cm} 
\smallskip \smallskip
Signal assignment.
\smallskip \smallskip
\end{minipage}\\

\end{tabular}
\end{center}

\begin{center}
\begin{tabular}{ c  c  c }
\multicolumn{3}{c}{\textbf{ATTRIBUTES OF SCALAR TYPES AND SUBTYPES}}\\ 
\hline \hline
VHDL-AMS & BON & Description \\ 
\hline \hline
\begin{minipage}[c]{2.4cm} 
\smallskip \smallskip
\lstinlinenc|T'left| \\
\lstinlinenc|T'right|\\
\lstinlinenc|T'high| \\
\lstinlinenc|T'low| \\
\lstinlinenc|T'ascending| \\
\lstinlinenc|T'image(x)| \\
\lstinlinenc|T'value(s)| \\
\lstinlinenc|T'pos(s)| \\
\lstinlinenc|T'val(x)| \\
\lstinlinenc|T'succ(x)| \\
\lstinlinenc|T'pred(x)| \\
\lstinlinenc|T'leftof(x)| \\
\lstinlinenc|T'rightof(x)| \\
\lstinlinenc|T'base| \\
\smallskip \smallskip
\end{minipage}  
&
\begin{minipage}[c]{4.6cm}
\centering 
\begin{lstlisting}[language=Bon]
deferred class ISCALTYPE[T]
 feature 
  left:T
  right:T
  low:T
  high:T
  image:STRING -> x:REAL
  value:STRING -> s:STRING
  pos:INTEGER -> s:STRING
  val:T  -> x:REAL
  succ:T -> x:REAL
  pred:T -> x:REAL
  leftof:T  -> x:REAL
  rightof:T -> x:REAL
  base:T
end
\end{lstlisting}
\smallskip \smallskip
\end{minipage}  
& 
\begin{minipage}[c]{6cm}  
\smallskip \smallskip
This class represents type interface 
with following features: first (leftmost) value,\xspace 
last (rightmost) value,\xspace 
least value,\xspace ascending range,\xspace 
value \textbf{x} of T,\xspace\textbf{s}
string presentation of value,\xspace
position number of x in T,\xspace
value at position x in T,\xspace
value at position one greater than x in T,\xspace
value at position one less than x in T,\xspace
value at position one to the left of x in T,\xspace
value at position one to the right of x in T,\xspace
base type of T, for use only as prefix of 
another attribute.
\smallskip \smallskip
\end{minipage}\\ 
\end{tabular}
\end{center}

\begin{center}
\begin{tabular}{ c  c  c }
\multicolumn{3}{c}{\textbf{TOLERANCE}}\\ 
\hline \hline
VHDL-AMS & BON & Description \\ 
\hline \hline
\begin{minipage}[c]{2.4cm} 
\centering
\smallskip \smallskip 
\smallskip \smallskip
\end{minipage}  
&
\begin{minipage}[c]{4.6cm}
\centering 
\smallskip \smallskip
\begin{lstlisting}[language=Bon]
deferred class IQTOLERANCE
 feature 
  checkAbsConv:BOOLEAN
  checkRelativeConv:BOOLEAN
  computeLTE:REAL
end
\end{lstlisting}
\smallskip \smallskip
\end{minipage}  
& 
\begin{minipage}[c]{6cm}  
\smallskip \smallskip
This class represents an quantity tolerance interface.
\smallskip \smallskip
\end{minipage}\\

\begin{minipage}[c]{2.4cm} 
\centering
\smallskip \smallskip 
\lstinlinenc|tolerance group|
\smallskip \smallskip
\end{minipage}  
&
\begin{minipage}[c]{4.6cm}
\centering 
\smallskip \smallskip
\begin{lstlisting}[language=Bon]
effective class TOLERANCE
 inherit IQTOLERANCE
 feature{NONE} 
  tol_name : STRING 
  tol_unit : STRING
  tol_dot : REAL
  tol_integ : REAL
 feature
  make 
    -> name,unit: STRING
    -> dot,integ: REAL
 feature 
  getName:STRING
  getUnit:STRING
  getDot:REAL
  getInteg:REAL   
end
\end{lstlisting}

\smallskip \smallskip
\end{minipage}  
& 
\begin{minipage}[c]{6cm}  
\smallskip \smallskip
The tolerance group of a subtype is a string value that 
may used by a model to group values determined by the quantities.
\smallskip \smallskip
\end{minipage}\\ 
\end{tabular}
\end{center}

\begin{center}
\begin{tabular}{ c  c  c }
\multicolumn{3}{c}{\textbf{EFFORT COMMON DOMAINS}}\\ 
\hline \hline
VHDL-AMS & BON & Description \\ 
\hline \hline

\begin{minipage}[c]{2.4cm} 
\centering
\smallskip \smallskip 
\lstinlinenc|voltage(V)|
\smallskip \smallskip
\end{minipage}  
&
\begin{minipage}[c]{4.6cm}
\centering 
\smallskip \smallskip
\begin{lstlisting}[language=Bon]
effective class VOLTAGE 
 inherit EFFORT
  feature tolerance:TOLERANCE
end  
\end{lstlisting}
\smallskip \smallskip
\end{minipage}  
& 
\begin{minipage}[c]{6cm}  
\smallskip \smallskip
Voltage effort for electrical domain.
\smallskip \smallskip
\end{minipage}\\ 

\begin{minipage}[c]{2.4cm} 
\centering
\smallskip \smallskip 
\lstinlinenc|velocity(V)|
\smallskip \smallskip
\end{minipage}  
&
\begin{minipage}[c]{4.6cm}
\centering 
\smallskip \smallskip
\begin{lstlisting}[language=Bon]
effective class VELOCITY
 inherit EFFORT
  feature tolerance:TOLERANCE
end 
\end{lstlisting}
\smallskip \smallskip
\end{minipage}  
& 
\begin{minipage}[c]{6cm}  
\smallskip \smallskip
Velocity effort for mechanical translation domain.
\smallskip \smallskip
\end{minipage}\\

\begin{minipage}[c]{2.4cm} 
\centering
\smallskip \smallskip 
\lstinlinenc|angular_velocity|
\smallskip \smallskip
\end{minipage}  
&
\begin{minipage}[c]{4.6cm}
\centering 
\smallskip \smallskip
\begin{lstlisting}[language=Bon]
effective class ANGULAR_VELOCITY
 inherit EFFORT
  feature tolerance:TOLERANCE
end 
\end{lstlisting}
\smallskip \smallskip
\end{minipage}  
& 
\begin{minipage}[c]{6cm}  
\smallskip \smallskip
Angular velocity effort for mechanical rotation domain.
\smallskip \smallskip
\end{minipage}\\


\begin{minipage}[c]{2.4cm} 
\centering
\smallskip \smallskip 
\lstinlinenc|pressure(P)|
\smallskip \smallskip
\end{minipage}  
&
\begin{minipage}[c]{4.6cm}
\centering 
\smallskip \smallskip
\begin{lstlisting}[language=Bon]
effective class PRESSURE
 inherit EFFORT
  feature tolerance:TOLERANCE
end 
\end{lstlisting}
\smallskip \smallskip
\end{minipage}  
& 
\begin{minipage}[c]{6cm}  
\smallskip \smallskip
Pressure effort for fluidic domain.
\smallskip \smallskip
\end{minipage}\\
\end{tabular}
\end{center}

\begin{center}
\begin{tabular}{ c  c  c }
\multicolumn{3}{c}{\textbf{FLOW COMMON DOMAINS}}\\ 
\hline \hline
VHDL-AMS & BON & Description \\ 
\hline \hline

\begin{minipage}[c]{2.4cm} 
\centering
\smallskip \smallskip 
\lstinlinenc|current(I)|
\smallskip \smallskip
\end{minipage}  
&
\begin{minipage}[c]{4.6cm}
\centering 
\smallskip \smallskip
\begin{lstlisting}[language=Bon]
effective class CURRENT 
 inherit EFFORT
  feature tolerance:TOLERANCE
end   
\end{lstlisting}
\smallskip \smallskip
\end{minipage}  
& 
\begin{minipage}[c]{6cm}  
\smallskip \smallskip
Current flow for electrical domain.
\smallskip \smallskip
\end{minipage}\\ 

\begin{minipage}[c]{2.4cm} 
\centering
\smallskip \smallskip 
\lstinlinenc|force(F)|
\smallskip \smallskip
\end{minipage}  
&
\begin{minipage}[c]{4.6cm}
\centering 
\smallskip \smallskip
\begin{lstlisting}[language=Bon]
effective class FORCE
 inherit EFFORT
  feature tolerance:TOLERANCE
end 
\end{lstlisting}
\smallskip \smallskip
\end{minipage}  
& 
\begin{minipage}[c]{6cm}  
\smallskip \smallskip
Force flow for mechanical translation domain.
\smallskip \smallskip
\end{minipage}\\

\begin{minipage}[c]{2.4cm} 
\centering
\smallskip \smallskip 
\lstinlinenc|torque(T)|
\smallskip \smallskip
\end{minipage}  
&
\begin{minipage}[c]{4.6cm}
\centering 
\smallskip \smallskip
\begin{lstlisting}[language=Bon]
effective class TORQUE
 inherit EFFORT 
  feature tolerance:TOLERANCE
end
\end{lstlisting}
\smallskip \smallskip
\end{minipage}  
& 
\begin{minipage}[c]{6cm}  
\smallskip \smallskip
Torque flow for mechanical rotation domain.
\smallskip \smallskip
\end{minipage}\\


\begin{minipage}[c]{2.4cm} 
\centering
\smallskip \smallskip 
\lstinlinenc|volumetric| \\
\lstinlinenc|flow_rate|
\smallskip \smallskip
\end{minipage}  
&
\begin{minipage}[c]{4.6cm}
\centering 
\smallskip \smallskip
\begin{lstlisting}[language=Bon]
effective class VOLUM_FLOW_RATE
 inherit EFFORT
  feature tolerance:TOLERANCE
end
\end{lstlisting}
\smallskip \smallskip
\end{minipage}  
& 
\begin{minipage}[c]{6cm}  
\smallskip \smallskip
Volumetric flow rate  for fluidic domain.
\smallskip \smallskip
\end{minipage}\\
\end{tabular}
\end{center}

\begin{center}
\begin{tabular}{ c c c }
\multicolumn{3}{ c }{\textbf{NATURE}}\\ 
\hline \hline
VHDL-AMS & BON & Description \\
\hline \hline
\multirow{3}{*}{
\begin{minipage}[c]{2.4cm} 
\centering  
\smallskip \smallskip
\lstinlinenc|nature|
\smallskip \smallskip
\end{minipage}
} 
&
\begin{minipage}[c]{4.6cm} 
\centering 
\smallskip \smallskip
\begin{lstlisting}[language=Bon]
class SNATURE
 inherit ISCALNATURE
 feature 
  ref:TERMINAL
end
\end{lstlisting}

\smallskip \smallskip
\end{minipage}
& 
\begin{minipage}[c]{6cm} 
\smallskip \smallskip
Scalar nature definition.
Reference terminal is created implicitly 
by nature declaration.  
\smallskip \smallskip
\end{minipage}\\
& 
\begin{minipage}[c]{4.6cm} 
\centering
\smallskip \smallskip
\begin{lstlisting}[language=Bon]
class ANATURE
 inherit
  IARRAY[SUBNATURE]
  feature
  make:ANATURE -> 
   array:SEQUENCE[SUBNATURE]  
end  
\end{lstlisting}

\smallskip \smallskip
\end{minipage}
& 
\begin{minipage}[c]{6cm} 
\smallskip \smallskip
Array nature definition.
\lstinlinen|ANATURE's| sequence 
contains \lstinlinen|SUBNATURE|. 
\smallskip \smallskip
\end{minipage}\\
& 
\begin{minipage}[c]{4.6cm} 
\centering 
\smallskip \smallskip
\begin{lstlisting}[language=Bon]
class RNATURE
  feature
   make:RNATURE -> 
    record:TABLE
end  
\end{lstlisting}
\smallskip \smallskip
\end{minipage} 
& 
\begin{minipage}[c]{6cm} 
\smallskip \smallskip
Record nature definition.
\lstinlinen|RNATURE's| map  
contains \lstinlinen|VALUE| as key elements.
The value elements must be either \lstinlinen|SUBNATURE| 
or \lstinlinen|IARRAY|.
\smallskip \smallskip
\end{minipage}\\ 

\begin{minipage}[c]{2.4cm} 
\centering 
\smallskip \smallskip
\lstinlinenc|subnature|
\smallskip \smallskip
\end{minipage}  
&
\begin{minipage}[c]{4.6cm}
\centering
\smallskip \smallskip
\begin{lstlisting}[language=Bon]
class SUBNATURE[TOLERANCE]
 inherit SNATURE
  feature tolerance:TOLERANCE  
\end{lstlisting}
\smallskip \smallskip
\end{minipage}  
& 
\begin{minipage}[c]{6cm}  
\smallskip \smallskip
Subnature definition.
\smallskip \smallskip
\end{minipage}\\
\end{tabular}
\end{center}

\begin{center}
\begin{tabular}{ c  c  c }
\multicolumn{3}{c}{\textbf{ATTRIBUTES OF SCALAR NATURES }}\\ 
\hline \hline
VHDL-AMS & BON & Description \\ 
\hline \hline

\begin{minipage}[c]{2.4cm} 
\smallskip \smallskip 
\lstinlinenc|N'across| \\ 
\lstinlinenc|N'through|
\smallskip \smallskip
\end{minipage}  
&
\begin{minipage}[c]{4.6cm}
\centering
\smallskip \smallskip 
\begin{lstlisting}[language=Bon]
deferred class ISCALNATURE
 feature 
  across:EFFORT
  through: FLOW
end   
\end{lstlisting}
\smallskip \smallskip
\end{minipage}  
& 
\begin{minipage}[c]{6cm}  
\smallskip \smallskip
The second group of predefined attributes gives 
information about the values of a scalar nature 
or subnature.
\end{minipage}\\  
\end{tabular}
\end{center}

\begin{center}
\begin{tabular}{ c  c  c }
\multicolumn{3}{c}{\textbf{ATTRIBUTES OF ARRAY TYPES AND NATURES}}\\ 
\hline \hline
VHDL-AMS & BON & Description \\ 
\hline \hline
\begin{minipage}[c]{2.4cm} 
\smallskip \smallskip
\lstinlinenc|A'left(n)| \\
\lstinlinenc|A'right(n)|\\
\lstinlinenc|A'high(n)| \\
\lstinlinenc|A'low(n)| \\
\lstinlinenc|A'range(n)| \\
\lstinlinenc|A'reverse_range(n)| \\
\lstinlinenc|A'length(n)| \\
\lstinlinenc|A'ascending(n)| \\
\smallskip \smallskip
\end{minipage}  
&
\begin{minipage}[c]{4.6cm}
\centering 
\begin{lstlisting}[language=Bon]
deferred class IARRAY[T]
 feature -> n:INTEGER
  left:T -> n:INTEGER
  right:T-> n:INTEGER
  low:T  -> n:INTEGER
  high:T -> n:INTEGER
  range:INTEGER     
         -> n:INTEGER
  reverse_range:INTEGER 
         -> n:INTEGER
  length:INTEGER 
         -> n:INTEGER
  ascending:BOOLEAN 
         -> n:INTEGER
end
\end{lstlisting}
\smallskip \smallskip
\end{minipage}  
& 
\begin{minipage}[c]{6cm}  
\smallskip \smallskip
This class represents array nature interface 
with following features: leftmost value in index range of dimension n,\xspace 
rightmost value in index range of dimension n,\xspace 
least value in index range of dimension n,\xspace 
greatest value in index range of dimension n,\xspace
index range of dimension n,\xspace
index range of dimension n reversed in direction and bounds,\xspace
length of index range of dimension n,\xspace and
ordering.\xspace
\smallskip \smallskip
\end{minipage}\\ 
\end{tabular}
\end{center}

\begin{center}
\begin{tabular}{ c  c  c }
\multicolumn{3}{c}{\textbf{STANDARD NATURES}}\\ 
\hline \hline
VHDL-AMS & BON & Description \\ 
\hline \hline
\begin{minipage}[c]{2.4cm} 
\centering 
\smallskip \smallskip 
\lstinlinenc|electrical|
\smallskip \smallskip
\end{minipage}  
&
\begin{minipage}[c]{4.6cm}
\centering
\smallskip \smallskip 
\begin{lstlisting}[language=Bon]
effective class ELECTRIC
 inherit SUBNATURE
end   
\end{lstlisting}
\smallskip \smallskip
\end{minipage}  
& 
\begin{minipage}[c]{6cm}  
\smallskip \smallskip
Electrical domain.
\end{minipage}\\ 


\begin{minipage}[c]{2.4cm} 
\centering 
\smallskip \smallskip 
\lstinlinenc|magnetic|
\smallskip \smallskip
\end{minipage}  
&
\begin{minipage}[c]{4.6cm}
\centering 
\smallskip \smallskip
\begin{lstlisting}[language=Bon]
effective class MAGNETIC
 inherit SUBNATURE
end   
\end{lstlisting}
\smallskip \smallskip
\end{minipage}  
& 
\begin{minipage}[c]{6cm} 
\smallskip \smallskip
Electromagnetic domain.
\smallskip \smallskip
\end{minipage}\\ 

\begin{minipage}[c]{2.4cm} 
\centering 
\smallskip \smallskip 
\lstinlinenc|translational|
\smallskip \smallskip
\end{minipage}  
&

\begin{minipage}[c]{4.6cm}
\centering
\smallskip \smallskip 
\begin{lstlisting}[language=Bon]
effective class TRANSLATION
 inherit SUBNATURE   
\end{lstlisting}
\smallskip \smallskip
\end{minipage}  
& 
\begin{minipage}[c]{6cm} 
\smallskip \smallskip 
Translational domain (linear displacement).
\smallskip \smallskip
\end{minipage}\\ 

\begin{minipage}[c]{2.4cm} 
\centering 
\smallskip \smallskip 
\lstinlinenc|translational_v|
\end{minipage}  
&
\begin{minipage}[c]{4.6cm}
\centering 
\smallskip \smallskip
\begin{lstlisting}[language=Bon]
effective class TRANS_VELOCITY
 inherit SUBNATURE
end   
\end{lstlisting}
\smallskip \smallskip
\end{minipage}  
& 
\begin{minipage}[c]{6cm} 
\smallskip \smallskip
Translational velocity domain.
\end{minipage}\\ 



\begin{minipage}[c]{2.4cm} 
\centering 
\smallskip \smallskip 
\lstinlinenc|rotational|
\end{minipage}  
&

\begin{minipage}[c]{4.6cm}
\centering
\smallskip \smallskip 
\begin{lstlisting}[language=Bon]
effective class ROTATION
 inherit SUBNATURE
end   
\end{lstlisting}
\smallskip \smallskip
\end{minipage}  
& 
\begin{minipage}[c]{6cm} 
\smallskip \smallskip
Rotational domain (angular displacement).
\smallskip \smallskip
\end{minipage}\\



\begin{minipage}[c]{2.4cm} 
\centering 
\smallskip \smallskip 
\lstinlinenc|rotational_v|
\end{minipage}  
&
\begin{minipage}[c]{4.6cm}
\centering
\smallskip \smallskip 
\begin{lstlisting}[language=Bon]
effective class ROT_VELOCITY
 inherit SUBNATURE
end   
\end{lstlisting}
\smallskip \smallskip
\end{minipage}  
& 
\begin{minipage}[c]{6cm} 
\smallskip \smallskip
Rotational velocity domain.
\smallskip \smallskip   
\end{minipage}\\ 



\begin{minipage}[c]{2.4cm} 
\centering 
\smallskip \smallskip 
\lstinlinenc|fluidic|
\end{minipage}  
&
\begin{minipage}[c]{4.6cm}
\centering
\smallskip \smallskip 
\begin{lstlisting}[language=Bon]
effective class FLUID
 inherit SUBNATURE
end   
\end{lstlisting}
\smallskip \smallskip
\end{minipage}  
& 
\begin{minipage}[c]{6cm} 
\smallskip \smallskip
Fluidic domain.
\smallskip \smallskip   
\end{minipage}\\ 

\begin{minipage}[c]{2.4cm} 
\centering 
\smallskip \smallskip 
\lstinlinenc|thermal|
\end{minipage} 
&
\begin{minipage}[c]{4.6cm}
\centering
\smallskip \smallskip 
\begin{lstlisting}[language=Bon]
effective class THERMAL
 inherit SUBNATURE
end   
\end{lstlisting}
\smallskip \smallskip
\end{minipage}  
& 
\begin{minipage}[c]{6cm} 
\smallskip \smallskip
Thermal domain.
\smallskip \smallskip   
\end{minipage}\\ 

\begin{minipage}[c]{2.4cm} 
\centering 
\smallskip \smallskip 
\lstinlinenc|radiant|
\smallskip \smallskip
\end{minipage}  
&
\begin{minipage}[c]{4.6cm}
\centering
\smallskip \smallskip 
\begin{lstlisting}[language=Bon]
effective class RADIANT
 inherit SUBNATURE
end   
\end{lstlisting}
\smallskip \smallskip
\end{minipage}  
& 
\begin{minipage}[c]{6cm} 
\smallskip \smallskip
Radiant domain.
\smallskip \smallskip   
\end{minipage}\\ 
\end{tabular}
\end{center}

\begin{center}
\begin{tabular}{ c  c  c }
\multicolumn{3}{c}{\textbf{TERMINAL ATTRIBUTES}}\\ 
\hline \hline
VHDL-AMS & BON & Description \\ 
\hline \hline

\begin{minipage}[c]{2.4cm} 
\smallskip \smallskip 
\lstinlinenc|T'reference| \\
\lstinlinenc|T'contribution|
\lstinlinenc|T'tolerance|
\smallskip \smallskip
\end{minipage}  
&
\begin{minipage}[c]{4.6cm}
\centering 
\smallskip \smallskip
\begin{lstlisting}[language=Bon]
deferred class ITERMINAL 
  feature 
   reference:EFFORT
   contribution:FLOW
   tolearnce:TOLERANCE
end  
\end{lstlisting}
\smallskip \smallskip
\end{minipage}  
& 
\begin{minipage}[c]{6cm}  
\smallskip \smallskip
This class represents an terminal interface with with 
following features:
\arcoss branch quantity from T to the reference terminal,\xspace  
\through quantity measuring the amount flowing through terminal,\xspace 
and tolerance group of terminal.
\smallskip \smallskip
\end{minipage}\\ 
\end{tabular}
\end{center}

\begin{center}
\begin{tabular}{ c  c  c }
\multicolumn{3}{c}{\textbf{TERMINAL}}\\ 
\hline \hline
VHDL-AMS & BON & Description \\ 
\hline \hline

\multirow{3}{*}{
\begin{minipage}[c]{2.4cm} 
\centering 
\smallskip \smallskip
\lstinlinenc|terminal|
\smallskip \smallskip
\end{minipage}
} 
&
\begin{minipage}[c]{4.6cm} 
\centering 
\smallskip \smallskip
\begin{lstlisting}[language=Bon]
effective class SCALAR_TERMINAL
 inherit ITERMINAL
 feature
  make:SCALAR_TERMINAL 
    -> nature:SUBNATURE  
end   
\end{lstlisting}
\smallskip \smallskip
\end{minipage}

& \begin{minipage}[l]{6cm} 
\smallskip \smallskip
Scalar terminal definition with 
reference and contribution which 
corresponds to the \arcoss and \through features derived from 
\lstinlinen|SUBNATURE|. 
\smallskip \smallskip
\end{minipage}\\
& \begin{minipage}[c]{4.6cm} 
\centering 
\smallskip \smallskip
\begin{lstlisting}[language=Bon]
effective class ARRAY_TERMINAL
 feature
  make:ARRAY_TERMINAL 
    -> nature:ANATURE 
end   
\end{lstlisting}
\smallskip \smallskip
\end{minipage}
& \begin{minipage}[c]{6cm} 
\smallskip \smallskip
Composite (array) terminal definition. 
\smallskip \smallskip
\end{minipage}\\

& \begin{minipage}[c]{4.6cm} 
\centering 
\smallskip \smallskip
\begin{lstlisting}[language=Bon]
effective class RECORD_TERMINAL
 feature
  make:RECORD_TERMINAL 
    -> nature:RNATURE 
end   
\end{lstlisting}
\smallskip \smallskip
\end{minipage} 
& \begin{minipage}[c]{6cm} 
\smallskip \smallskip
Composite (record) terminal  definition. 
\smallskip \smallskip
\end{minipage}\\ 
\end{tabular}
\end{center}

\begin{center}
\begin{tabular}{ c  c  c }
\multicolumn{3}{c}{\textbf{ATTRIBUTES OF QUANTITY}}\\ 
\hline \hline
VHDL-AMS & BON & Description \\ 
\hline \hline

\begin{minipage}[l]{2.4cm} 
\smallskip \smallskip 
\lstinlinenc|Q'tolerance|  \\
\lstinlinenc|Q'above(E)|   \\
\lstinlinenc|Q'delayed(t)| \\
\lstinlinenc|Q'dot|        \\
\lstinlinenc|Q'integ|      \\
\lstinlinenc|Q'ztf| \\
\lstinlinenc|(num,den,t| \\
\lstinlinenc|initial_delay)|       \\
\lstinlinenc|Q'ltf| \\
\lstinlinenc|(num,den)| \\
\lstinlinenc|Q'zoh| \\
\lstinlinenc|(t,initial_delay)|       \\
\lstinlinenc|Q'slew|       \\
\lstinlinenc|(max_rising_slope,| \\
\lstinlinenc|max_falling_slope)| \\
\smallskip \smallskip
\end{minipage}  
&
\begin{minipage}[c]{4.6cm}
\centering 
\smallskip \smallskip
\begin{lstlisting}[language=Bon]
deferred class IQUANTITY
 feature 
  tolerance:TOLERANCE
  above:BOOLEAN
   -> e:REAL
  delayed:REAL
   -> t:UNIVERSAL_TIME
  dot:FREE_QUANTITY 
  integ:FREE_QUANTITY
  ztf:FREE_QUANTITY
   -> num:SEQUENCE[REAL]
   -> den:SEQUENCE[REAL]
   -> t:UNIVERSAL_TIME
   -> initial_delay:UNIVERSAL_TIME 
  ltf:FREE_QUANTITY
   -> num:SEQUENCE[REAL]
   -> den:SEQUENCE[REAL] 
  zoh:FREE_QUANTITY
   -> t:UNIVERSAL_TIME
   -> initial_delay:UNIVERSAL_TIME  
  slew:FREE_QUANTITY
   -> max_rising_slope:REAL
   -> max_falling_slope:REAL
end
\end{lstlisting}
\smallskip \smallskip
\end{minipage}  
& 
\begin{minipage}[c]{6cm}  
\smallskip \smallskip
This calss represents quantity interface with following
attributes: \\
Tolerance group string presentation for scalar quantity.
Above attribute is true if the value of Q is grater than 
the value of e, otherwise false. \\
Delayed attribute represents value of quantity delayed by T.NOW-T.\\
Dot attribute represents quantity with the same type 
as quantity Q whose value is the \textbf{derivative} 
of Q with respect to time.\\
Integ attribute represents quantity with the same type 
as quantity Q whose value is the \textbf{integral} 
of Q with respect to time. \\
Ztf quantity represents \emph{z-domain} transfer function of Q.\\
Ltf quantity represents \emph{Laplace-domain} transfer function of Q. \\
Zoh quantity represents \emph{zero order hold} of Q.\\
Slow attribute represents quantity whose value follows that of Q, 
but whose derivative with respect to time is limited by the 
slopes.
\smallskip \smallskip
\end{minipage}\\ 
\end{tabular}
\end{center}

\begin{center}
\begin{tabular}{ c  c  c }
\multicolumn{3}{c}{\textbf{QUANTITY}}\\ 
\hline \hline
VHDL-AMS & BON & Description \\ 
\hline \hline
\multirow{3}{*}{
\begin{minipage}[c]{2.4cm} 
\centering 
\smallskip \smallskip
\lstinlinenc|quantity|
\smallskip \smallskip
\end{minipage}
} 
&
\begin{minipage}[c]{4.6cm} 
\centering 
\smallskip \smallskip
\begin{lstlisting}[language=Bon]
effective class FREE_QUANTITY
 inherit IQUANTITY
  feature
  make:FREE_QUANTITY
   -> aType:SUBTYPE
end 
\end{lstlisting}
\smallskip \smallskip
\end{minipage}
& 
\begin{minipage}[l]{6cm} 
\smallskip \smallskip
Free quantity represents an analog-valued
object used in signal-flow modeling. 
\smallskip \smallskip
\end{minipage}\\
& 
\begin{minipage}[c]{4.6cm} 
\centering 
\smallskip \smallskip
\begin{lstlisting}[language=Bon]
effective class BRANCH_QUANTITY
  feature 
    make:BRANCH_QUANTITY
    through:FLOW
    across:EFFORT
    terminalFrom:SEQUENCE[ITERMINAL]
    terminalTo:SEQUENCE[ITERMINAL]
end 
\end{lstlisting}
\smallskip \smallskip
\end{minipage}
& 
\begin{minipage}[c]{6cm} 
\smallskip \smallskip
Branch quantity.
\begin{itemize}
\item \textbf{accross}:
Represents the difference in \emph{effort} 
between terminals.
Effort is represented by the \textbf{accross} terminal's subnature.
\item \textbf{through}:
Represents the \emph{flow} through the branch between
terminals.
Flow is represented by the \textbf{through} of terminal's subnature.
\end{itemize}
 
\smallskip \smallskip
\end{minipage}\\
& 
\begin{minipage}[c]{4.6cm} 
\centering 
\smallskip \smallskip
\begin{lstlisting}[language=Bon]
effective class SOURCE_QUANTITY
  feature 
    make:SOURCE_QUANTITY
end 
\end{lstlisting}
\smallskip \smallskip
\end{minipage} 
& 
\begin{minipage}[c]{6cm} 
\smallskip \smallskip
Source quantity. 
\smallskip \smallskip
\end{minipage}\\
\end{tabular}
\end{center}




\begin{center}
\begin{tabular}{ c  c  c }
\multicolumn{3}{c}{\textbf{QUANTITY PORT}}\\ 
\hline \hline
VHDL-AMS & BON & Description \\ 
\hline \hline

\begin{minipage}[c]{2.4cm} 
\centering
\smallskip \smallskip 

\smallskip \smallskip
\end{minipage}  
&
\begin{minipage}[c]{4.6cm}
\centering 
\smallskip \smallskip

\smallskip \smallskip
\end{minipage}  
& 
\begin{minipage}[c]{6cm}  
\smallskip \smallskip
definition.
\smallskip \smallskip
\end{minipage}\\ 
\end{tabular}
\end{center}

\begin{center}
\begin{tabular}{ c  c  c }
\multicolumn{3}{c}{\textbf{TERMINAL PORT}}\\ 
\hline \hline
VHDL-AMS & BON & Description \\ 
\hline \hline

\begin{minipage}[c]{2.4cm} 
\centering
\smallskip \smallskip 

\smallskip \smallskip
\end{minipage}  
&
\begin{minipage}[c]{4.6cm}
\centering 
\smallskip \smallskip

\smallskip \smallskip
\end{minipage}  
& 
\begin{minipage}[c]{6cm}  
\smallskip \smallskip
definition.
\smallskip \smallskip
\end{minipage}\\ 
\end{tabular}
\end{center}

\begin{center}
\begin{tabular}{ c  c  c }
\multicolumn{3}{c}{\textbf{SIGNAL}}\\ 
\hline \hline
VHDL-AMS & BON & Description \\ 
\hline \hline

\begin{minipage}[c]{2.4cm} 
\centering
\smallskip \smallskip 

\smallskip \smallskip
\end{minipage}  
&
\begin{minipage}[c]{4.6cm}
\centering 
\smallskip \smallskip

\smallskip \smallskip
\end{minipage}  
& 
\begin{minipage}[c]{6cm}  
\smallskip \smallskip
definition.
\smallskip \smallskip
\end{minipage}\\ 
\end{tabular}
\end{center}

\begin{center}
\begin{tabular}{ c  c  c }
\multicolumn{3}{c}{\textbf{ATTRIBUTES OF SIGNAL}}\\ 
\hline \hline
VHDL-AMS & BON & Description \\ 
\hline \hline

\begin{minipage}[c]{2.4cm} 
\centering
\smallskip \smallskip 

\smallskip \smallskip
\end{minipage}  
&
\begin{minipage}[c]{4.6cm}
\centering 
\smallskip \smallskip

\smallskip \smallskip
\end{minipage}  
& 
\begin{minipage}[c]{6cm}  
\smallskip \smallskip
definition.
\smallskip \smallskip
\end{minipage}\\ 
\end{tabular}
\end{center}

\begin{center}
\begin{tabular}{ c  c  c }
\multicolumn{3}{c}{\textbf{WAIT}}\\ 
\hline \hline
VHDL-AMS & BON & Description \\ 
\hline \hline

\begin{minipage}[c]{2.4cm} 
\centering
\smallskip \smallskip 

\smallskip \smallskip
\end{minipage}  
&
\begin{minipage}[c]{4.6cm}
\centering 
\smallskip \smallskip

\smallskip \smallskip
\end{minipage}  
& 
\begin{minipage}[c]{6cm}  
\smallskip \smallskip
definition.
\smallskip \smallskip
\end{minipage}\\ 
\end{tabular}
\end{center}

%======================================================================

\end{document}

%%% Local Variables: 
%%% mode: latex
%%% eval: 
%%% TeX-master: t
%%% End: 
