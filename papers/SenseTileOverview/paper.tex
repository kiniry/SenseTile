%
% "The UCD CASL SenseTile System", for IEEE Pervasive Computing
% Joseph R. Kiniry, Vieri del Bianco, Dragan Stosic
% $Id: paper.tex 2795 2007-09-23 19:35:53Z dmz $
%

\documentclass[times,10pt,twocolumn]{article} 
\usepackage{latex8}
\usepackage{times}

\usepackage{ifpdf}
\usepackage{a4wide}
\usepackage{pdfsync}

\ifpdf
\usepackage[pdftex]{graphicx}
\else
\usepackage{graphicx}
\fi

%\usepackage{todonotes}

% \usepackage{html}
% \usepackage{url}
\usepackage{xspace}
%\usepackage{doublespace}
\usepackage{tabularx}
\usepackage{epsfig}
\usepackage{amsmath}
\usepackage{amsfonts}
\usepackage{amssymb}
\usepackage{eucal}
\usepackage{stmaryrd}
% \ifpdf
% \usepackage[centredisplay]{diagrams}
% \else
% \usepackage[centredisplay,PostScript=dvips]{diagrams}
% \fi
\usepackage{float}

\ifpdf
\usepackage[pdftex,bookmarks=false,a4paper=false,
            plainpages=false,naturalnames=true,
            colorlinks=true,pdfstartview=FitV,
            linkcolor=blue,citecolor=blue,urlcolor=blue,
            pdfauthor="Joseph R. Kiniry and Vieri del Bianco and Dragan Stosic"]{hyperref}
\else
\usepackage[dvips,linkcolor=blue,citecolor=blue,urlcolor=blue]{hyperref}
\fi

\newcommand{\tablesize}{\footnotesize}
\newcommand{\eg}{e.g.,\xspace}
\newcommand{\ie}{i.e.,\xspace}
\newcommand{\etc}{etc.\xspace}
\newcommand{\myhref}[2]{\ifpdf\href{#1}{#2}\else\htmladdnormallinkfoot{#2}{#1}\fi}
%\newcommand{\myhref}[2]{\emph{#2}}
\newcommand{\todo}{\textbf{TODO:}}
\newcommand{\ST}{\emph{SenseTile}\xspace}
\newcommand{\STs}{\emph{SenseTiles}\xspace}
\newcommand{\datastore}{\STs Scientific Datastore\xspace}
\newcommand{\computefarm}{The \STs Scientific Compute Farm\xspace}
\newcommand{\computefarmlong}{UCD CASL \STs Software and Data Compute Server Farm\xspace}
\newcommand{\sensorfarm}{The \STs\xspace}

%---------------------------------------------------------------------
% New commands, macros, \etc
%---------------------------------------------------------------------

%% \input{kt}

%=====================================================================

\begin{document}

\title{The UCD CASL SenseTile System}

\author{Joseph R. Kiniry, Vieri del Bianco, Dragan Stosic\\
UCD CASL: Complex and Adaptive Systems Laboratory and\\
School of Computer Science and Informatics,\\
University College Dublin,\\
Belfield, Dublin 4, Ireland,\\
kiniry@acm.org, vieri.delbianco@gmail.com, dragan.stosic@gmail.com\\
}

\maketitle

%======================================================================
%\thispagestyle{empty}
\begin{abstract}
  
\todo{\emph{We probably have only 8 pages, in the IEEE format, if this is going to
\myhref{http://ieeexplore.ieee.org/xpl/RecentIssue.jsp?punumber=7756}{IEEE
  Pervasive Computing}}.}

The \textbf{UCD CASL SenseTile System} is a large-scale, general-purpose
sensor system installed at the University College Dublin in Dublin,
Ireland.  \textbf{Our facility provides a capability---unique in the world in
terms of its scale and flexibility---for performing in-depth
investigation of both the specific and general issues in large-scale
sensor networking.}

This system integrates a sensor platform, a datastore, and a compute
farm.  The sensor platform is a custom-designed, but inexpensive,
sensor platform (called the \ST) paired with general-purpose
small-scale compute nodes, including everything from low-power PDAs to
powerful touchscreen-based portable computers.  The datastore is a
multi-terabyte scale scientific datastore into which sensor data
flows, and in which online and offline scientific computation of
sensor and other scientific data takes place.  The compute farm is a
large-scale Linux, Solaris, and OS~X-based compute farm used by
scientists for these online and offline scientific computations.

This system has several novel features: (1) new deployments of \STs
requires zero configuration; (2) \STs have several unusual built-in
sensors and emitters, (3) sensor nodes can be extended via a
general-purpose USB bus, (4) \STs contain an FPGA that is dynamically
updatable on a per-experiment basis, and (5) configuring, deploying,
and managing experiments using \STs is performed by scientists via
either programmatic APIs or via a declarative graphical or textual
specification language.

\end{abstract}

%======================================================================
\Section{Introduction}

Sensor networks are widely seen as an essential technology for
supporting next-generation scientific and engineering challenges,
including environmental monitoring, climate change, assisted living,
national security and intensive agriculture. Addressing these problems
requires two complementary strands of research:
\begin{enumerate}
\item to understand the \emph{specific physical and/or social
    phenomena of interest} against which to fit data collection and
  analysis; and
\item to understand the \emph{general scientific and technological
    principles governing scalable sensor networking} in order to
  maximise the leverage gained from on-going developments.
\end{enumerate}
 
Amongst the core general challenges are the collection of large
volumes of multimedia data, its storage, cataloging, retrieval and
processing in such a way as to minimise physical and intellectual
costs of accessing the available data. This is especially important in
pursuit of cross-disciplinary projects involving different analysis
methodologies and constraints. At the same time, it is impossible to
study sensor networking completely separate from practical
applications which provide real-world validation and verification of
the techniques being developed.

This paper describes a facility, the \ST system, to support
large-scale experiments with complex multimedia sensing and processing
at terabyte scales.  The key piece of equipment we have designed and
fabricated is a rich and reusable sensor platform, called the \ST,
that is easily deployed into the built environment for
experimentation.  This particular piece of hardware is representative
of platforms suitable for wider uses, including on-board data storage
and processing capacity and associated high-speed interconnect.

The key research challenges this system supports includes:
\begin{itemize}
  \setlength{\itemsep}{1pt}
  \setlength{\parskip}{0pt}
  \setlength{\parsep}{0pt}
\item dealing with uncertainty in sensor systems,
\item robust management of large (multi-terabyte) sensing data on
  heterogeneous platforms,
\item interaction and activity recognition,
\item sparse signal analysis, and
\item coding and securing sensor interactions.
\end{itemize}

Underpinning each of these key challenges is a fundamental need for an
integrated, multi-modal sensor testbed that has two unique
characteristics. Firstly, it must be \emph{reconfigurable and be
  easily targeted} to collect defined data sets.  Secondly---and
perhaps most importantly---it must be \emph{of sufficient size and
  complexity} to ensure that the challenges listed above, and the
resulting theories and techniques, can be evaluated in a
scientifically rigorous manner.  It is our contention that \emph{our
  facility provides a capability---unique in the world in terms of its
  scale and flexibility---for performing in-depth investigation of
  both the specific and general issues in large-scale sensor
  networking.}  Furthermore, a synergistic benefit is the support for
the development of reference data sets and services for use by other
researchers internationally.

%=====================================================================
\Section{Conclusion}

A conclusion.

%======================================================================
%% \nocite{ex1,ex2}
\bibliographystyle{latex8}
\bibliography{extra,%
              abbrev,%
              ads,%
              category,%
              complexity,%
              hypertext,%
              icsr,%
              knowledge,%
              languages,%
              linguistics,%
              meta,%
              metrics,%
              misc,%
              modeling,%
              modeltheory,%
              reuse,%
              rewriting,%
              softeng,%
              specification,%
              ssr,%
              technology,%
              theory,%
              web,%
              upcoming,%
              upcoming_conferences,%
              conferences,%
              workshops,%
              verification,%
              escjava,%
              jml,%
              nijmegen}

%======================================================================
% Fin

\end{document}

%%% Local Variables: 
%%% mode: latex
%%% eval: 
%%% TeX-master: t
%%% End: 
