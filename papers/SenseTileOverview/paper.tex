%
% "The UCD CASL SenseTile System", for IEEE Pervasive Computing
% Joseph R. Kiniry, Vieri del Bianco, Dragan Stosic
% $Id: paper.tex 2795 2007-09-23 19:35:53Z dmz $
%

\documentclass[times,10pt,twocolumn]{article} 
\usepackage{latex8}
\usepackage{times}

\usepackage{ifpdf}
\usepackage{a4wide}
\usepackage{pdfsync}

\ifpdf
\usepackage[pdftex]{graphicx}
\else
\usepackage{graphicx}
\fi

%\usepackage{todonotes}

% \usepackage{html}
% \usepackage{url}
\usepackage{xspace}
%\usepackage{doublespace}
\usepackage{tabularx}
\usepackage{epsfig}
\usepackage{amsmath}
\usepackage{amsfonts}
\usepackage{amssymb}
\usepackage{eucal}
\usepackage{stmaryrd}
% \ifpdf
% \usepackage[centredisplay]{diagrams}
% \else
% \usepackage[centredisplay,PostScript=dvips]{diagrams}
% \fi
\usepackage{float}

\ifpdf
\usepackage[pdftex,bookmarks=false,a4paper=false,
            plainpages=false,naturalnames=true,
            colorlinks=true,pdfstartview=FitV,
            linkcolor=blue,citecolor=blue,urlcolor=blue]{hyperref}
\else
\usepackage[dvips,linkcolor=blue,citecolor=blue,urlcolor=blue]{hyperref}
\fi

\newcommand{\tablesize}{\footnotesize}
\newcommand{\eg}{e.g.,\xspace}
\newcommand{\ie}{i.e.,\xspace}
\newcommand{\etc}{etc.\xspace}
\newcommand{\myhref}[2]{\ifpdf\href{#1}{#2}\else\htmladdnormallinkfoot{#2}{#1}\fi}
%\newcommand{\myhref}[2]{\emph{#2}}
\newcommand{\todo}{\textbf{TODO:}}
\newcommand{\ST}{\emph{SenseTile}\xspace}
\newcommand{\STs}{\emph{SenseTiles}\xspace}
\newcommand{\datastore}{\STs Scientific Datastore\xspace}
\newcommand{\computefarm}{The \STs Scientific Compute Farm\xspace}
\newcommand{\computefarmlong}{UCD CASL \STs Software and Data Compute Server Farm\xspace}
\newcommand{\sensorfarm}{The \STs\xspace}

%---------------------------------------------------------------------
% New commands, macros, \etc
%---------------------------------------------------------------------

%% \input{kt}

%=====================================================================

\begin{document}

\title{The UCD CASL SenseTile System}

\author{\ldots}

\maketitle

%======================================================================
%\thispagestyle{empty}
\begin{abstract}

The \textbf{\ST System} is a large-scale, general-purpose
sensor system installed at the University College Dublin in Dublin,
Ireland. 
Our facility provides services for performing in-depth
investigation of aspects and issues related to large-scale sensor 
networking.

This system integrates heterogenous hardware and software systems. 
The hardware systems are composed by a sensors layer (1000+ sensors), a gateway 
layer (100+ minipc), an High-performance computing cluster (50+ cores) and a datastore (100+ TB). 
The software systems integrated in the platform comprehend: (1) a set of
drivers to communicate with the severeal sensors systems deployed, (2)
a distributed sensor data stream management to enable data streaming between 
the processor nodes and to the datastore, (3) a data provenance and transformation
distributed service to enable the search and inspection of the meta data associated to the sensor streams, 
(4) a processor nodes monitoring and management system to ease deployment and mass 
installation of software and configuration on processor nodes, (5) a scientific platform
to process and inspect the data collected, batch or interactive computation are possible, using stand alone 
servers or using the HPC dedicated cluster.

The \ST Sensor Board is part of the sensors layer, and is a custom-designed, but inexpensive,
sensor board, paired directly with a processor node through an USB channel.

The \ST system exhibits the following novel aspects in sensor management: 
(1) multi layer architecture, to keep separate sensors, gateways, HPC processing cluster and the datastore;
(2) integrated management of a large number of gateways collecting data from several heterogenous sources;
(3) the integration of several heterogeneous software systems, to support sensor, data streams, gateways and 
datastore management, and scientifical experiments execution.

\end{abstract}



\Section{Introduction}

Sensor networks are widely seen as an essential technology for supporting next-generation scientific and engineering
challenges, including environmental monitoring, climate change, assisted living, national security and intensive
agriculture. 
Addressing these problems requires two complementary strands of research:
(1) understand the specific physical and social phenomena of interest against which to fit data collection and analysis;
(2) understand the general scientific and technological principles governing scalable sensor networking in order to 
maximise the leverage gained from on-going developments.

There are several technological challanges to be faced to solve these problems: (1) the collection, storage and management 
of large volumes of multimedia data and the meta data associated; (2) the interactive and offline setup and monitoring of
the sensors; (3) the interactive and offline setup of live data collection experiments (data streaming, and soft real time data 
processing); (4) the post processing analysis and visualization of the data collected.

A vertical approach in sensor data collection and analysis, which is the nowadays dominant approach, is not 
able to grasp the general problem.
Each research community has its own requirements and constraints on data collection and analysis, and considering the 
problem only from a single perspective would limit the usefulness of the tools developed, the data collected and the data analysis.
When the problems addressed are too specific, too focused on a single discipline, the solutions found are hardly 
reusable in different areas.
Thus, the sensor management platforms and the data repositories designed and developed, and, most importantly, the data 
collected are ineffective and inappropriate, if not completely unusable and useless, when they are 
tried to be exported and reused in even slightly dissimilar domains from the original one.

\TODO{provide examples, provide references}

Merging requirements from heterogeneous disciplines is a challanging task; not only because different domains generates 
different languages creating a de facto barrier in communication and reciprocal understanding \TODO{provide references}, but 
also because big tradeoffs are needed to be made to accomodate most of the requirements. 
The tradeoff, to be sensible, needs to be chosen directly by the involved research communities and domain experts, and 
this is can be solved only involving them directly in cross-disciplinary projects.

Sensor data collection and analysis, to be fully understood and exploited, and to make the solutions envisioned general 
enough to be reusable, need to be studied and addressed in cross-disciplinary projects involving different analysis 
methodologies and constraints. 
Cross-disciplinary projects are irremissible in the development of general solutions and tools to sensor data acquisition, 
storage and analysis.



heterogeneous domains involved ->
some domains are far away from computer science ->
ease of use of tools, techniques, data ->
increased probability of tools, techniques, data to be succesfully used






\begin{itemize}
  \setlength{\itemsep}{1pt}
  \setlength{\parskip}{0pt}
  \setlength{\parsep}{0pt}
\item dealing with uncertainty in sensor systems,
\item robust management of large (multi-terabyte) sensing data on
  heterogeneous platforms,
\item interaction and activity recognition,
\item sparse signal analysis, and
\item coding and securing sensor interactions.
\end{itemize}


%=====================================================================
\Section{Conclusion}

A conclusion.

%======================================================================
%% \nocite{ex1,ex2}
\bibliographystyle{latex8}
\bibliography{extra,%
              abbrev,%
              ads,%
              category,%
              complexity,%
              hypertext,%
              icsr,%
              knowledge,%
              languages,%
              linguistics,%
              meta,%
              metrics,%
              misc,%
              modeling,%
              modeltheory,%
              reuse,%
              rewriting,%
              softeng,%
              specification,%
              ssr,%
              technology,%
              theory,%
              web,%
              upcoming,%
              upcoming_conferences,%
              conferences,%
              workshops,%
              verification,%
              escjava,%
              jml,%
              nijmegen}

%======================================================================
% Fin

\end{document}

%%% Local Variables: 
%%% mode: latex
%%% eval: 
%%% TeX-master: t
%%% End: 
