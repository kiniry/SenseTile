\documentclass[]{final_report}
\usepackage{graphicx}
\usepackage{hyperref}


%%%%%%%%%%%%%%%%%%%%%%
%%% Input project details
\def\studentname{Ciaran Palmer}
\def\projecttitle{SensorML on SenseTile}
\def\supervisorname{Your Supervisor Name}
\def\moderatorname{Your Moderator Name}


\begin{document}

\maketitle
\tableofcontents\pdfbookmark[0]{Table of Contents}{toc}\newpage

%%%%%%%%%%%%%%%%%%%%%%
%%% Your Abstract here

\begin{abstract}
SenseTileSensor Board

SensorML description.

WebServer to access sensor observations

SensorML Bon Mapping with a tool and method to develop sensorML descriptions

\end{abstract}




\newpage


%%%%%%%%%%%%%%%%%%%%%%
%%% Acknowledgments

\chapter*{Acknowledgments}


%%%%%%%%%%%%%%%%%%%%%%
%%% Introduction

\chapter{Introduction}

The aim of this thesis project was to prototype a Web Service for accessing sensor observations from the SenseTile Sensor Board. SenseTile is a sensor and processing package including motion sensors, RFID sensors, temperature sensor, audio sensors, pressure sensors, light level sensors video sensors among others. It is used as a replacement for standard ceiling tiles to provide smart building services as part of a Web Sensor Network.

Descriptions of the SenseTile Sensor Board were to be developed using Sensor Model Language (SensorML)\cite{SensoMLref}. SensorML is a language that describes the process of sensor measurement by providing standard models and XML encoding for such processes. SensorML was be used as it is becoming a standard for describing sensor data processing as part of the SWE[ref].

A  SensorML to BON Mapping was also to be explored . BON\cite{BONref} is a notation and a method for Object Oriented systems analysis and design. Based on this mapping it was proposed to develop a tool/method to automatically generate SensorML descriptions. Generally SensorML is hand generated with all the shortcomings of such an approach. The proposed approach would allow sensor processing to defined formally in BON and then refined to JML/Java. 


\chapter{ Background Research}

How was problem solved before.


\chapter{SensorML}
\section{SensorML Overview}
The Open Geospatial Consortium (OGC) Sensor Web Enablement (SWE) [ref]  is establishing interfaces and protocols that will enable a standerised “Sensor Web”. It is hoped that an easier integration and development of sensor applications across sensor networks will be facilitated by this standarising activity. Sensor Modelling Language (SensorML) is one part of the SWE initiative and is used to describe sensor systems and the processing of observations from sensor systems.

The purposes of SensorML are to:
• Provide descriptions of sensors and sensor systems for inventory management
• Provide sensor and process information in support of resource and observation discovery
• Support the processing and analysis of the sensor observations
• Support the geolocation of observed values (measured data)
• Provide performance characteristics (e.g., accuracy, threshold, etc.)
• Provide an explicit description of the process by which an observation was obtained (i.e., it’s lineage)
• Provide an executable process chain for deriving new data products on demand (i.e., derivable observation)
• Archive fundamental properties and assumptions regarding sensor systems

SensorML provides a functional model of sensors and an XML encoding to describe sensors and their observations.
Sensors are modelled as processes that convert real phenomena to data. Processes take inputs, process them and result in one or several outputs. They can also define relevant metadata. The processes can be connected together in chains using SensorML ProcessChains or Systems. 

When Modelling a sensor system in SensorML the main entities used are Systems and Components. A SensorML System is used to group sensors that are related spatially to one another.  A Component is an atomic process that generally converts a physical phenomen to a digital number. ProcessModels are another modelling that is used to describe a pure process that has no physicality

Quantity, Count, Boolean, Category, and Time provide the basic primitives. Within SensorML, these serve as the basis for specifying all inputs, outputs, and parameters within a Process

DataRecord or DataArray.

Detector model.

The SensorML description of the SenseTile Sensor Board is shown in the following section.

\section{SensorML Description of SenseTile}

SensorML System and Component were used to describe the SenseTile system and it's sensors. The sensor specifications were used to generate the metadata.
The Sensors are modelled a detectors. They are process that take scalar values
and can perform transformations on them.

System is providing the mapping of inputs to processes and the outputs.

It is further recognized that most sensor observations consist of at least two processes: a sampling process and a conversion process.
The 
 \begin{figure}
\includegraphics{SensorML_SenseTile_System_design_tuple_pa3.png}
\caption{Block diagram of SenseTile System process}\label{fig:SensorML_SenseTile_design.png}
\end{figure}

\subsection{SenseTile System}
\subsection{Light Sensor}
\subsection{Thermistor}
\subsection{Pressure Sensor}
\subsection{Accelerometer}
\subsection{Acoustic Sensor}

Types
 - SWE -

Count - BigInteger
Quantity - Double

 * Carries an array of int primitives.
 * All data is casted to other types when requested.

public class DataBlockInt extends AbstractDataBlock


\newpage
\section{BON to SensorML mapping}

\chapter{Design}
High level design
\section{SenseTile Web Service}

DataProducer - use PacketInput stream to access SenseTile observation packets.

Steps in the process these sensorml components are roughly.

System reads packets and averages them for 1 second.

Thermistor components reads out the temperature value
and its output is sent to both the raw data and the celcius
converter component and on to the tuples.

These tuples are sent to webservice provider as observations.


DataProvider - provides an RMI interface to the DataProducer to update. Provides a Web Service interface  based on Sensor Observation Service. Not fully as is a very complex interface.

\section{SensorML generation tool}

SensorML to BON mapping

BON to Java

Instantiate Java classes with data

Walk Java class and generate sensorml.



\chapter{ Detailed Design and Implementation}

\section{SenseTile WebService}

DataProvider is based on AXIS 2.0/Spring

DataProducer POJO/VastLib/UCD SenseTile Driver Lib

\chapter{Results}

Tested against real SenseTile and sensor data retrieved by a testclient.


\chapter{ Conclusions and Future Work}

what has been achieved
the weaknesses of your approach

\chapter{References}



\chapter{Appendices}


\newpage
\begin{thebibliography}{99}
\bibitem{SensoMLref}Open Geospatial Consortium Inc., OpenGIS Sensor Model Language (SensorML) Implementation Specification, 2007
\bibitem{BONref}Kim Waldén and Jean-Marc Nerson , "Seamless Object-Oriented Software Architecture", 1995
\end{thebibliography}
\label{endpage}



\end{document}

\end{article}
